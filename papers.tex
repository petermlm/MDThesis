\section{Initial Papers in Detail}

This section contain long comments about the first papers which were explored
for this project.

% =============================================================================
\subsection{\texttt{IND\_001} - Data Mining Algorithms to Classify Students}

Paper is~\cite{ind_001}. Lists a lot of resources for \gls{dm} in \gls{lms}.

\subsubsection{Comment}

This paper presents the results of a few experiments made to Moodle data. It
also presents a data mining tool for Moodle called Moodle Data Mining Tool.
Finally, the papers shows some results on the accuracy of some algorithms.

The first section, after the introduction, talks about the background of data
mining in \gls{lms}. It goes on describing a few different tasks and citing
other works in the area. The second section describes the Moodle Data Mining
Tool. The third section presents a few experiments and their results. The
experiments intend on evaluating the performance of some algorithms. The
algorithms are executed with data from 7 courses of the Moodle of Cordoba
University.

This papers is very useful for it gives a few resources on data mining in
\gls{lms}. The resources state different techniques and models used in Moodle
Data, and from here it is possible to explore similar works.

The experiments are not very detailed. A few points are made on resampling, for
the data has a clear imbalance. Then the accuracy of some algorithms executed
with the University's data is shown. It is visible that the accuracy of every
algorithm is not better then 65\%, which is very strange.

A lot of data mining methods and algorithms are listed. The experiments don't
really go into detail, and the Moodle Data Mining Tool doesn't play a big role
in the experiments or in the conclusion. Nevertheless, it is a great paper that
points to useful resources.

\subsubsection{Relevant Points}

What is taken from this paper is the argument that ``(\dots) one of the most
useful~\gls{dm} tasks in e-learning is classification''. The paper then lists a
few sources where~\gls{edm} has been used. They are the following:

\begin{itemize}
    \item Discover potential student groups with similar characteristics and
        reactions to a particular pedagogical strategy;~\cite{ind_002}
    \item Detect students' misuse or game-playing; (2)
    \item Group students who are hint-driven or failure-driven and find common
        misconceptions that students possess; (34)
    \item Identify learners with low motivation and find remedial actions to
        lower drop-out rates; (9)
    \item Predict/classify students when using intelligent tutoring systems;
        (16)
\end{itemize}

In detail, there are methods that have been applied to detect students marks.

\begin{itemize}
    \item Predicting students' grades from test scores using neural
        networks;~\cite{ind_003}
    \item Predicting student academic success (classes that are successful or
        not) using discriminant function analysis; (21)
    \item Classifying students using genetic algorithms to predict their final
        grade; (30)
    \item Predicting a student's academic success (to classify as low, medium
        and high risk classes) using different data mining methods; (30)
    \item Predicting a student’s marks (pass and fail classes) using regression
        techniques in Hellenic Open University data; (18)
    \item Using neural network models from Moodle logs; (11)
\end{itemize}

\subsubsection{Work From This Paper}

From this paper we will analyse the listed techniques and methodologies.

% =============================================================================
\subsection{\texttt{IND\_002} - Discovering Decision Knowledge From Web Log
Portfolio For Managing Classroom Processes By Applying Decision Tree And Data
Cube Technology}

Paper is~\cite{ind_002}. An experiment in which web logs of a \gls{lms} get
analysed in order to find good student behaviour. This paper was found
from~\cite{ind_001}.

\subsubsection{Comment}

\gls{lms} systems, like Moodle, generate logs which describe that activities of
it's users. Those logs contain entries for when students login, when they ask
questions, when they offer proposals on so on. Analysing these logs in order to
understand if student performance is good or bad is not trivial because they
were not made for pedagogical purposes. So some~\gls{dm} techniques are used to
gain an understanding of student performance.

The paper states that the data from the logs is initially transformed to be
inserted into a relational database. SQL is then used to pre-process that data.
Data cube technology is used for multidimensional data viewing. Decision trees
are also constructed from the data. These decision trees constructed from the
web logs are the final results which shed light on student behaviour.

The logs in questions are made in a system that implements IBIS. The logs are
processed and inserted into data cubes for later analysis. The aim of the
analysis is to check if certain pedagogical strategies work. The analysis later
use decision trees.

The paper is too long and repeats a few sections. It is not that clear how the
logs are processed or how SQL and data cubes are used for analysus. It is also
not clear how decision trees are built or what they represent.

This is a useful paper with some different methods worth reading about, even if
data cube isn't that common in the rest of the literature explored until this
point.

\subsubsection{Relevant Points}

This paper is useful because it is an example of mining web logs for~\gls{edm}.
Also useful for it shows the usage of data cubes, which isn't that common.

\subsubsection{Work From This Paper}

No further work is done for this paper.

% =============================================================================
\subsection{\texttt{IND\_003} - Predicting Performance From Test Scores Using
Backpropagation and Counterpropagation}

Paper is~\cite{ind_003}. Neural Networks are used to predict students grades
from placement test responses. This paper was found from~\cite{ind_001}.

\subsubsection{Comment}

Backpropagation and Counterpropagation are two types of neural networks that
are trained in different ways. This paper uses both types of networks to
predict students grades. The data set used in this paper is small, but more
data is generated from it. The data refers to placement tests for a course on
Calculus I at West University Boulevard in Florida.

The initial data set is small. It contains 116 entries. Each entry has 25
features, which although not specified in the paper, are stated to be results
from placement tests. Each entry also contains the final grade of the student,
from \textit{A} to \textit{D} stating a different level of approval, and
\textit{F} when a student failed. Because the data is small, some data has been
generated.

The networks contains 25 inputs, for the features, and 5 outputs, for each
grade. An output of \textit{A} will be a vector of the form $ [1, -1, -1, -1,
-1] $, for \textit{B} it will be $ [-1, 1, -1, -1, -1] $, and so on. The
networks contain a single hidden layer which number of units varies for
different trainings. The number of units are 15, 20, 25, and 30 for different
tests.

It is shown how the networks are trained and are able to achieve correctly
classify students, given their input data. It seems that an optimal number of
units in the hidden layer is 25. Some networks achieve 80\% of correctly
classified instances for a pure dataset. Training networks with generated
datasets is yields greater correctly classified percentages between 90\% and
100\%.

Although the results shown are very good, the paper doesn't go into detail
about the inputs of the networks. It is important to know to format of the data
of the students placement tests, how the data was pre-processed, and into what
was it pre-processed. Like this, the paper leaves to much to speculation. It
would also be nice to see how these methods are applied in other courses.

The paper presents the hypothesis that neural networks are good for predicting
student's performance based on existing placement tests for a given course on
Calculus I. It then shows how data is taken and used to generate more data. The
networks are trained and are able to achieve great results.

\subsubsection{Relevant Points}

This paper is important because it is prof that neural networks can be used to
classify student performance. It is also important because it shows a useful
and successful tasks in the world of~\gls{edm}.

Ideas from this paper may be used for later study, namely how data is divided
into fold for training and testing, how resampling is done to generate more
data, and how it also influences the correctness of the trained models.

\subsubsection{Work From This Paper}

This paper only has four references, and all of them are related with Neural
Networks. Those references might came in handy if work is to be done in Neural
Networks, but not for data mining in~\gls{lms}.

% =============================================================================
\subsection{\texttt{IND\_004} - Web Usage Mining for Predicting Final Marks of
Students That Use Moodle Courses}

Paper is~\cite{ind_004}. Paper~\cite{ind_001} is related to this one, for the
authors and experiments in this paper are the same as in~\cite{ind_001}, but
are presented with much more detail.

\subsubsection{Reading Notes}

\Gls{lms} are being widely used. Moodle is very famous.

\Gls{lms} store a lot of data regarding student profiles, their grades, when
they login, etc. Analysing this data is hard to do by hand. Data Mining
techniques may be used.

Data Mining has been applied to a variety of problems in e-learning. Like:

\begin{itemize}
    \item Dealing with assessment of students' learning performance.
    \item Providing course adaptations and learning recommendations.
    \item etc. See more at 17
\end{itemize}

The objective of this paper is to classify students with similar final marks
into different groups depending on the activities carried out in a web-based
course.

Data from Cordoba University is used in 7 Moodle engineering courses. The data
comes from three online activities:

\begin{itemize}
    \item Quizzes
    \item Assignments
    \item Forums
\end{itemize}

The data was divided into 10 pairs of training and test data files for
stratified 10-fold cross-validation (reference 81).

The first experiment consists on executing every algorithm with three datasets
and getting the geometric mean:

\begin{itemize}
    \item First dataset. All 438 instances/students with all features.
    \item Second dataset. Filtered instances where students didn't complete
        every Moodle activity. Got 135 instances.
    \item Third dataset. All instance but only with the 4 features which were
        selected using a few methods.
\end{itemize}

The second experiment is basically the same as the first, but the data ha two
pre-processing tasks applied to it: discretization and rebalancing.

Paper also gives some background on data mining. Also talks about Moodle Data
Mining Tool.

\subsubsection{Comment}

This paper mines web logs of~\gls{lms} in order to take meaning from them. Like
in~\cite{ind_001} and~\cite{ind_002}, it is asserted that extracting meaningful
information from logs is not trivial, so~\gls{dm} methodologies are used. The
paper makes it so, while presenting the various algorithms testing with
different dataset, having some of which being the results of resampling.

The paper shows how data from seven courses is taken and mined. That data
consists of web logs detailing the activities of students in Moodle. Those
activities are, for example, when a student logged in, when it answered a
question, when it saw a forum post, etc. Several algorithms are executed over
this data in order to classify students. Further pre-processing tasks are
executed over the same data and differences in accuracy are observed.

We are able to see the results of various algorithms executed with different
datasets. It is possible to observe how different pre-processing methodologies
influence the final results. The paper presents several ways to pre-process
data, and many algorithms.

Looking at all of the results, we see that no algorithm got a better ``Global
Percentage of Correctly Classified'' percentage than 66\%. This level of
accuracy leaves much to be desired. The geometric mean is used to evaluate the
accuracy of algorithms with imbalanced data. We see that many algorithms get
0\% in this indicator, while other got percentages below 50\% and close to 0\%.
This is very bad.

Decision trees are concluded to be the best model for they perform better in
term of ``Global Percentage of Correctly Classified'' and in geometric mean.
These models are best executed with the presented methods for pre-processing.
The final values are still considered to be low, but the results make sense.

\subsubsection{Relevant Points}

Again, we see how using data mining on~\gls{lms} logs is useful. We also see
how classification is useful.

This paper presents a few tools which will be extremely important to
pre-process data in order to prepare it for the algorithms. It is intend that
decision trees are a good model to classify students into different groups
given their activity in Moodle.

\subsubsection{Work From This Paper}

Find more about Moodle Data Mining Tool and find more work done by these
authors.

Look further into using decision trees to mine Moodle data.

% =============================================================================
\subsection{\texttt{IND\_005} - Using Educational Data Mining Methods to Study the
Impact of Virtual Classroom in E-Learning}

Paper is~\cite{ind_005}. I haven't seen this paper has a reference of the
previous ones. But found it on Google Scholar while searching for Moodle Data
Mining. The techniques used in this paper are similar to~\cite{ind_002,
ind_004}.

\subsubsection{Reading Notes}

The paper starts by talking about difficulties in e-learning environments. Such
as students feeling lonely and unsupported. They also talk about how web logs
need to be mined so meaningful information is taken from them.

This study is done to web logs. They have 824 students in 11 courses, but claim
that the number is superior. The study is done only to students who are remote.
They say that other studies such as 8, 13, and 14 e-learning platforms was used
to facilitate teacher-student interactions.

They start by preprocessing a group of features of students. Then they do
feature ranking using a few algorithms and build a decision tree based on the
Gain Ratio metric.

The decision tree predicts students grades based on their features.

\subsubsection{Comment}

This paper is another~\gls{dm} analysis made to Moodle. Like in previous works,
such as~\cite{ind_002, ind_004}, this paper goes through the process of
pre-processing Moodle data from e-learning courses, applying~\gls{dm}
techniques, and interpreting the results. The objectives of this paper are to
predict student grades given their usage of Moodle in an e-learning
environment.

The origins of the data are shown to come from 824 students in 11 courses. The
features of the data come from logs which say when a student logged in, when it
asked a question, and so on. The data mining process consists on applying a few
feature selection algorithms, such as gain ratio, $ \chi^2 $, SVN, and others,
to rank the features. In the end, a decision tree is built from the ranking
using gain ratio.

The built tree is able to predict the grades of a student given their features.
The first two levels of a built tree are shown and some interpretation is
given. For example, assertions are made about students that participate less
then 11 times in virtual classrooms which will probably fail the exams.

Although the results are presented and seem logical, the probabilities given
for each assertion of the decision tree are made so with small probabilities.
The previous example, for instance, only has 55\% probability. Some ways to
evaluate the accuracy of the built tree should be taken.

The methods listed in this paper are well explained and the results are clear.
Building decisions trees to predict students grades is a sound objective and a
common one in~\gls{edm}. The only negative points are the fact that the built
tree is not evaluated.

\subsubsection{Relevant Points}

This paper is very relevant because it displays similar techniques
to~\cite{ind_002, ind_004}. The techniques for pre-processing and mining, the
datasets, and the conclusions are all very similar and will doubtless be very
important as a reference in future work.

\subsubsection{Work From This Paper}

References 18, 16, 5, 8, are 9 worth looking into.

% =============================================================================
\subsection{\texttt{IND\_006} - Predicting students' marks from Moodle logs using
neural network models}

Paper is~\cite{ind_006}. But found it on Google Scholar while searching for
Moodle Data Mining. This is another paper that uses neural networks
like~\cite{ind_003} did.

\subsubsection{Comment}

Another paper that attempts to predict student's success given their usage of
Moodle. In this case we see the usage of neural networks, as we've seen
in~\cite{ind_003}. Like in~\cite{ind_001, ind_002, ind_004, ind_005}, Moodle
logs are processed alongside final grades in order to build features for
students.  Neural networks are then trained using that data.

The usual information is taken from logs. For example, date and time of access
of Moodle, if a user saw an item, if it wrote somethings in a forum etc. This
data is pre-processed in order to build a dataset. The built dataset for these
experiments is made from entries of 240 students. The trained neural networks
are Radial Basis Functions. They are trained in a few different ways.

It is shown how training a neural network with an incremental hidden layer
algorithm yields a model capable of achieving 80\% of correctly classified
instances. Other ways to train the networks are shown to be very successful
achieving percentages above 70\%.

Not much is known about how the success metric is calculated, or how the
alternative ways to train the network actually work. But they are listed
nonetheless. The paper leaves a few questions opened, namely the usefulness of
the actual study.

Neural networks are presented as a good model to predict students success, and
are shown to be, in fact, a great model for it. Different ways to train a
network are shown and some hints as to how data is pre-processed into useful
datasets are also shown. Even if the paper is short, doesn't go into detail
about certain aspects of the project, and leaves things opened, it is still a
great resource.

\subsubsection{Relevant Points}

Networks networks are once again shown to work in predicting student's success
given their Moodle usage habits.

\subsubsection{Work From This Paper}

See reference 7 in this paper about Radial Basis Functions.

% =============================================================================
\subsection{\texttt{IND\_007} - Educational data mining: A survey from 1995 to
2005}

Paper is~\cite{ind_007}. This is an overview of~\gls{edm}.

Starts by comparing e-learning Data Mining to e-commerce. States that the
differences are as follows:

\begin{itemize}
    \item Domain: E-commerce's purpose is to guide clients in purchasing.
        E-learning's purpose is to guide students in learning.
    \item Data: E-learning has more data then logs, because we have student's
        information.
    \item Objective: The objective of e-commerce is to maximise profit. In
        e-learning is to improve learning, which is more subjective.
    \item Technique: Techniques are different.
\end{itemize}

EDM may be oriented towards different actors who take different points of view
and interest from the analysis. Those actors are:

\begin{itemize}
    \item Students, who want to improve their studying method.
    \item Educators, who want feedback on their education methods, and who want
        to classify students or group their needs and other features.
    \item Oriented towards academics responsible and administrators, who want
        to optimize site performance, among other indicators.
\end{itemize}

The paper lists types of e-learning systems~(\gls{lms}) and how their logs are
made. Also states a few limitations on those logs.

The paper lists data pre-processing tasks and techniques that will also be
useful.

% =============================================================================
\subsection{\texttt{IND\_008} - Educational Data Mining: A Review of the State
of the Art}

Paper is~\cite{ind_008}.

This paper is a very useful overview of the~\gls{edm} field. It goes on to
explain several points about this field.

The paper asserts that instrumental educational software and databases
containing student information provide big repositories of data which can be
studied to obtain meaning from them. Those studies are made using Statistical
Analysis and Data Mining methods.

EDM has emerged from the need to implement Data Mining methods over these
repositories of data. \gls{edm} as now been subdivided into three areas:

\begin{itemize}
    \item Offline education
    \item E-Learning and~\gls{lms}
    \item Intelligent tutoring systems (ITS) and Adaptive Educational
        Hypermedia Systems (AEHS)
\end{itemize}

An important point is that some pre-processing method and some post-processing
methods for data visualization are not necessarily~\gls{dm} topics, but are
closely related, therefore, they are mentioned in this paper. The mentioned
techniques for pre and post-processing are relevant in the~\gls{edm} field.

The following is a list of Stakeholders of~\gls{edm} projects. It is very
important that this is listed in the final thesis text because it provides the
whole motivation for having done a project like this in the first place.

\begin{itemize}
    \item Learners/Students
    \item Educators/Teachers
    \item Course Developers/Educational Researchers
    \item Learning Providers/Universities
    \item School Administrators and similar
\end{itemize}

For Learners. The results of can be used to recommend activities to students
and suggest interesting learning experiments which are alternative to the
experiences of a particular student. These results can also be used to
recommend interesting courses or point to useful discussions.

For Educators. The results may be used to get objective feedback about
instruction and analyse students. It becomes possible to do the following:

\begin{itemize}
    \item Find common mistakes in students
    \item Predict student grades based on student activities
    \item Determine effective activities
\end{itemize}

For Course Developers. Evaluating the effectiveness of a particular course, how
it is structured, what activities are the most useful and effective, are
important points to take into account when making changes to a course.

For Learning Providers. These are the entities that actually provide the means
to implement learning. These are, for example, Universities, both public and
private, and similar learning institutions.~\gls{edm} can give a powerful
overview of what is the most cost-effective way of assuring good grades,
retention of student, and addition of students who will do well in university.

For School Administrators.~\gls{edm} can give insight on the best ways to
organize human and material resources. In the context of this
project,~\gls{edm} is useful to evaluate the effectiveness of~\gls{edm}.

The paper states that publications in~\gls{edm} have grown a lot and gives some
number for it. There are also a list of publication that are considered to be
extremely relevant.

% =============================================================================
\subsection{\texttt{IND\_009} - Evolutionary algorithms for subgroup discovery
applied to e-learning data}

Paper is~\cite{ind_009}. Applying evolutionary algorithms to find association
rules.

Association rules are rules that link events together following the form, ``If
event $ A $ happened, the event $ B $ is likely to occur''. These are fuzzy
rules, because an event $ A $ happening doesn't necessarily imply that event $
B $ will always happen, rather, the rules give a probabilistic notion that
should be understood has, if $ A $ happened, then $ B $ will probably also
happen.

The paper talks about subgroups in the context of association rules and genetic
algorithms to learn rules from data. The classic association rules learning
algorithms, which are Apriori-SD and CN2-SD, are mentioned.

Like other papers from these authors, the usual Moodle data is shown and
several algorithms are applied over it. The results for those algorithms is
presented and discussed. The results seemed to be quite good.
