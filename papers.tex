\chapter{Papers}

% =============================================================================
\section{\texttt{IND\_007} - Data Mining Algorithms to Classify Students}

Paper is~\cite{ind_007}. Lists a lot of resources for \gls{dm} in \gls{lms}.

\subsection{Comment}

This paper presents the results of a few experiments made to Moodle data. It
also presents a data mining tool for Moodle called Moodle Data Mining Tool.
Finally, the papers shows some results on the accuracy of some algorithms.

The first section, after the introduction, talks about the background of data
mining in \gls{lms}. It goes on describing a few different tasks and citing
other works in the area. The second section describes the Moodle Data Mining
Tool. The third section presents a few experiments and their results. The
experiments intend on evaluating the performance of some algorithms. The
algorithms are executed with data from 7 courses of the Moodle of Cordoba
University.

This papers is very useful for it gives a few resources on data mining in
\gls{lms}. The resources state different techniques and models used in Moodle
Data, and from here it is possible to explore similar works.

The experiments are not very detailed. A few points are made on resampling, for
the data has a clear imbalance. Then the accuracy of some algorithms executed
with the University's data is shown. It is visible that the accuracy of every
algorithm is not better then 65\%, which is very strange.

A lot of data mining methods and algorithms are listed. The experiments don't
really go into detail, and the Moodle Data Mining Tool doesn't play a big role
in the experiments or in the conclusion. Nevertheless, it is a great paper that
points to useful resources.

\subsection{Relevant Points}

What is taken from this paper is the argument that ``(\dots) one of the most
useful~\gls{dm} tasks in e-learning is classification''. The paper then lists a
few sources where~\gls{edm} has been used. They are the following:

\begin{itemize}
    \item Discover potential student groups with similar characteristics and
        reactions to a particular pedagogical strategy;~\cite{ind_008}
    \item Detect students' misuse or game-playing; \TODO{Yet to be explore}
    \item Group students who are hint-driven or failure-driven and find common
        misconceptions that students possess; \TODO{Yet to be explore}
    \item Identify learners with low motivation and find remedial actions to
        lower drop-out rates; \TODO{Yet to be explore}
    \item Predict/classify students when using intelligent tutoring systems;
        \TODO{Yet to be explore}
\end{itemize}

In detail, there are methods that have been applied to detect students marks.

\begin{itemize}
    \item Predicting students' grades from test scores using neural
        networks;~\cite{ind_009}
    \item Predicting student academic success (classes that are successful or
        not) using discriminant function analysis; \TODO{Yet to be explore}
    \item Classifying students using genetic algorithms to predict their final
        grade; \TODO{Yet to be explore}
    \item Predicting a student's academic success (to classify as low, medium
        and high risk classes) using different data mining methods; \TODO{Yet
        to be explore}
    \item Predicting a student’s marks (pass and fail classes) using regression
        techniques in Hellenic Open University data; \TODO{Yet to be explore}
    \item Using neural network models from Moodle logs; \TODO{Yet to be
        explore}
\end{itemize}

\subsection{Work From This Paper}

From this paper we will analyse the listed techniques and methodologies.

% =============================================================================
\section{\texttt{IND\_008} - Discovering Decision Knowledge From Web Log
Portfolio For Managing Classroom Processes By Applying Decision Tree And Data
Cube Technology}

Paper is~\cite{ind_008}. An experiment in which web logs of a \gls{lms} get
analysed in order to find good student behaviour. This paper was found
from~\cite{ind_007}.

\subsection{Comment}

\gls{lms} systems, like Moodle, generate logs which describe that activities of
it's users. Those logs contain entries for when students login, when they ask
questions, when they offer proposals on so on. Analysing these logs in order to
understand if student performance is good or bad is not trivial because they
were not made for pedagogical purposes. So some~\gls{dm} techniques are used to
gain an understanding of student performance.

The paper states that the data from the logs is initially transformed to be
inserted into a relational database. SQL is then used to pre-process that data.
Data cube technology is used for multidimensional data viewing. Decision trees
are also constructed from the data. These decision trees constructed from the
web logs are the final results which shed light on student behaviour.

The logs in questions are made in a system that implements IBIS. The logs are
processed and inserted into data cubes for later analysis. The aim of the
analysis is to check if certain pedagogical strategies work. The analysis later
use decision trees.

The paper is too long and repeats a few sections. It is not that clear how the
logs are processed or how SQL and data cubes are used for analysus. It is also
not clear how decision trees are built or what they represent.

This is a useful paper with some different methods worth reading about, even if
data cube isn't that common in the rest of the literature explored until this
point.

\subsection{Relevant Points}

This paper is useful because it is an example of mining web logs for~\gls{edm}.
Also useful for it shows the usage of data cubes, which isn't that common.

\subsection{Work From This Paper}

No further work is done for this paper.

% =============================================================================
\section{\texttt{IND\_009} - Predicting Performance From Test Scores Using
Backpropagation and Counterpropagation}

Paper is~\cite{ind_009}. Neural Networks are used to predict students grades
from placement test responses. This paper was found from~\cite{ind_007}.

\subsection{Comment}

Backpropagation and Counterpropagation are two types of neural networks that
are trained in different ways. This paper uses both types of networks to
predict students grades. The data set used in this paper is small, but more
data is generated from it. The data refers to placement tests for a course on
Calculus I at West University Boulevard in Florida.

The initial data set is small. It contains 116 entries. Each entry has 25
features, which although not specified in the paper, are stated to be results
from placement tests. Each entry also contains the final grade of the student,
from \textit{A} to \textit{D} stating a different level of approval, and
\textit{F} when a student failed. Because the data is small, some data has been
generated.

The networks contains 25 inputs, for the features, and 5 outputs, for each
grade. An output of \textit{A} will be a vector of the form $ [1, -1, -1, -1,
-1] $, for \textit{B} it will be $ [-1, 1, -1, -1, -1] $, and so on. The
networks contain a single hidden layer which number of units varies for
different trainings. The number of units are 15, 20, 25, and 30 for different
tests.

It is shown how the networks are trained and are able to achieve correctly
classify students, given their input data. It seems that an optimal number of
units in the hidden layer is 25. Some networks achieve 80\% of correctly
classified instances for a pure dataset. Training networks with generated
datasets is yields greater correctly classified percentages between 90\% and
100\%.

Although the results shown are very good, the paper doesn't go into detail
about the inputs of the networks. It is important to know to format of the data
of the students placement tests, how the data was pre-processed, and into what
was it pre-processed. Like this, the paper leaves to much to speculation. It
would also be nice to see how these methods are applied in other courses.

The paper presents the hypothesis that neural networks are good for predicting
student's performance based on existing placement tests for a given course on
Calculus I. It then shows how data is taken and used to generate more data. The
networks are trained and are able to achieve great results.

\subsection{Relevant Points}

This paper is important because it is prof that neural networks can be used to
classify student performance. It is also important because it shows a useful
and successful tasks in the world of~\gls{edm}.

Ideas from this paper may be used for later study, namely how data is divided
into fold for training and testing, how resampling is done to generate more
data, and how it also influences the correctness of the trained models.

\subsection{Work From This Paper}

This paper only has four references, and all of them are related with Neural
Networks. Those references might came in handy if work is to be done in Neural
Networks, but not for data mining in~\gls{lms}.

% =============================================================================
\section{\texttt{IND\_003} - Web Usage Mining for Predicting Final Marks of
Students That Use Moodle Courses}

Paper is~\cite{ind_003}. Paper~\cite{ind_007} is related to this one, for the
authors and experiments in this paper are the same as in~\cite{ind_007}, but
are presented with much more detail.

\subsection{Comment}

This paper mines web logs of~\gls{lms} in order to take meaning from them. Like
in~\cite{ind_007} and~\cite{ind_008}, it is asserted that extracting meaningful
information from logs is not trivial, so~\gls{dm} methodologies are used. The
paper makes it so, while presenting the various algorithms testing with
different dataset, having some of which being the results of resampling.

The paper shows how data from seven courses is taken and mined. That data
consists of web logs detailing the activities of students in Moodle. Those
activities are, for example, when a student logged in, when it answered a
question, when it saw a forum post, etc. Several algorithms are executed over
this data in order to classify students. Further pre-processing tasks are
executed over the same data and differences in accuracy are observed.

We are able to see the results of various algorithms executed with different
datasets. It is possible to observe how different pre-processing methodologies
influence the final results. The paper presents several ways to pre-process
data, and many algorithms.

Looking at all of the results, we see that no algorithm got a better ``Global
Percentage of Correctly Classified'' percentage than 66\%. This level of
accuracy leaves much to be desired. The geometric mean is used to evaluate the
accuracy of algorithms with imbalanced data. We see that many algorithms get
0\% in this indicator, while other got percentages below 50\% and close to 0\%.
This is very bad.

Decision trees are concluded to be the best model for they perform better in
term of ``Global Percentage of Correctly Classified'' and in geometric mean.
These models are best executed with the presented methods for pre-processing.
The final values are still considered to be low, but the results make sense.

\subsection{Relevant Points}

Again, we see how using data mining on~\gls{lms} logs is useful. We also see
how classification is useful.

This paper presents a few tools which will be extremely important to
pre-process data in order to prepare it for the algorithms. It is intend that
decision trees are a good model to classify students into different groups
given their activity in Moodle.

\subsection{Work From This Paper}

Find more about Moodle Data Mining Tool and find more work done by these
authors.

Look further into using decision trees to mine Moodle data.
