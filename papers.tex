\chapter{Papers}

% =============================================================================
\section{\texttt{IND\_007} - Data Mining Algorithms to Classify Students}

Paper is \cite{ind_007}. Lists a lot of resources for other DM in \gls{lms}.

\subsection{Comment}

This paper presents the results of a few experiments made to Moodle data. It
also presents a data mining tool for Moodle called Moodle Data Mining Tool.
Finally, the papers shows some results on the accuracy of some algorithms.

The first section, after the introduction, talks about the background of data
mining in \gls{lms}. It goes on describing a few different tasks and citing
other works in the area. The second section describes the Moodle Data Mining
Tool. The third section presents a few experiments and their results. The
experiments intend on evaluating the performance of some algorithms. The
algorithms are executed with data from 7 courses of the Moodle of Cordoba
University.

This papers is very useful for it gives a few resources on data mining in
\gls{lms}. The resources state different techniques and models used in Moodle
Data, and from here it is possible to explore similar works.

The experiments are not very detailed. A few points are made on resampling, for
the data has a clear imbalance. Then the accuracy of some algorithms executed
with the University's data is shown. It is visible that the accuracy of every
algorithm is not better then 65\%, which is very strange.

A lot of data mining methods and algorithms are listed. The experiments don't
really go into detail, and the Moodle Data Mining Tool doesn't play a big role
in the experiments or in the conclusion. Nevertheless, it is a great paper that
points to useful resources.

\subsection{Relevant Points}

What is taken from this paper is the argument that ``(\dots) one of the most
useful DM tasks in e-learning is classification''.

\begin{itemize}
    \item Discover potential student groups with similar characteristics and
        reactions to a particular pedagogical strategy; \TODO{Yet to be
        explore}
    \item Detect students' misuse or game-playing; \TODO{Yet to be explore}
    \item Group students who are hint-driven or failure-driven and find common
        misconceptions that students possess; \TODO{Yet to be explore}
    \item Identify learners with low motivation and find remedial actions to
        lower drop-out rates; \TODO{Yet to be explore}
    \item Predict/classify students when using intelligent tutoring systems;
        \TODO{Yet to be explore}
\end{itemize}

In detail, there are methods that have been applied to detect students marks.

\begin{itemize}
    \item Predicting students' grades from test scores using neural networks;
        \TODO{Yet to be explore}
    \item Predicting student academic success (classes that are successful or
        not) using discriminant function analysis; \TODO{Yet to be explore}
    \item Classifying students using genetic algorithms to predict their final
        grade; \TODO{Yet to be explore}
    \item Predicting a student's academic success (to classify as low, medium
        and high risk classes) using different data mining methods; \TODO{Yet
        to be explore}
    \item Predicting a student’s marks (pass and fail classes) using regression
        techniques in Hellenic Open University data; \TODO{Yet to be explore}
    \item Using neural network models from Moodle logs; \TODO{Yet to be
        explore}
\end{itemize}

\subsection{Work From This Paper}

From this paper we will analyse the listed techniques and methodologies.

% =============================================================================
\section{\texttt{IND\_008} - Discovering decision knowledge from web log
portfolio for managing classroom processes by applying decision tree and data
cube technology}

\WORKON{Need to finish this paper's comment}

Paper is~\cite{ind_008}. An experiment in which web logs of a \gls{lms} get
analysed in order to find good student behaviour. This paper was found
from~\cite{ind_007}.

\subsection{Comment}

\gls{lms} systems, like Moodle, generate logs which describe that activities of
it's users. Those logs contain entries for when students login, when they ask
questions, when they offer proposals on so on. Analysing these logs in order to
understand if student performance is good or bad is not trivial. So
some~\gls{dm} techniques are used to gain an understanding of student
performance.

The paper states that the data from the logs is initially transformed to be
inserted into a relational database. SQL is then used to pre-process that data.
Data cube technology is used for multidimensional data viewing. Decision trees
are also constructed from the data.

\subsection{Relevant Points}

\TODO{Write this section}

\subsection{Work From This Paper}

\TODO{Write this section}
