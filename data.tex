\chapter{Data}

% =============================================================================
\section{Data Exploration}

To begin with, we explore what was initially provided in terms of data. The
repository is a spreadsheet book with 12 sheets. The sheets contain raw data
and some calculated fields.

In order to explore the data using the bash tools (cut, grep, etc) and tools
like Python with Pandas, they were exported to a CSV format. Some
transformations may be applied to these files in other to clean useless points
or make the data easier to process.

To begin, each sheet is analysed individually. Conclusions on each list are
registered in the fist section of this chapter.

\subsection{Sheet 1 - \texttt{UCs\_SIIUE\_Moodle.csv}}

This sheet lists a few courses (referred to in their native name as
\textit{Unidades Curriculares}). Some courses have a Moodle page while others
do not.

The sheets contain four non blank columns. The first and second column are a
list of courses represented has their Course Id and Moodle Id, which is
different.

The third column is just a calculated copy of the first. It doesn't add
anything and it might have been a loose processing step at some point.

The fourth column contains a list of course with no Moodle and therefore no
Moodle Id.

Only the following columns are relevant:

\begin{itemize}
    \item \texttt{Cod UC} - Course's code
    \item \texttt{No moodle} - Course's code in the Moodle
\end{itemize}

\subsection{Sheet 2 - \texttt{complete\_rate.csv}}

This sheet contain a listing of the completion rate of each course. The list
has tree columns. A column for course id in the Moodle (\texttt{c\_moodleid}),
the number of students in that course (\texttt{n\_students}), and the
completion rate of those courses (\texttt{complete\_rate}).

So a course with 100 students having a completion rate of 75\% means that 75
students actually completed that course.

\subsection{Sheet 3 - \texttt{completed.csv}}

Contains a listing of which student was approved at which course. The sheet has
three columns. The first column is the course id of the Moodle
(\texttt{c\_moodleid}), the second is the user id of the Moodle
(\texttt{u\_moodleid}), and the third is a column (\texttt{completed}) which
has the value 1 if the student completed that course, or 0 if not.

The user id may be assigned to a professor, but in this sheet every user id is
of a student.

\subsection{Sheet 4 - \texttt{Notas UCS SIIUE.csv}}

There is a lot of information in this sheet.

\TODO{Explore}

\subsection{Sheet 5 - \texttt{Profile}}

This sheet contains some analysis made to the data. To begin with, there are
the following columns:

\begin{itemize}
    \item \texttt{name\_c} - Name of the course
    \item \texttt{name\_s} - Name of the student
    \item \texttt{id\_c} - Id of the course
    \item \texttt{n\_resrc\_c} - Number of resources in the course
    \item \texttt{n\_actv\_c} - Number of activities in the course
    \item \texttt{id\_s} - Id of the student
    \item \texttt{n\_cons\_s} - \TODO{}
    \item \texttt{n\_subm\_s} - \TODO{}
    \item \texttt{result\_s} - Results of that student in that course
\end{itemize}

In order for a student to get approved, some creteria must be followed.

\subsection{Sheet 6 - \texttt{ResultadosUCMoodle}}

This sheet contains a lot of information about the courses. To begin with, we
get a relation between the three ways of identifying a course in the data. A
course is identified by the Course's Code (\texttt{Cod Siiue}), Course's Code
in the Moodle (\texttt{Cod Moodle}), or by a course id (\texttt{course}). The
course id is an integer number, while the codes contain information about the
department.

The department (\texttt{Dep}) is calculate from the course's code (\texttt{Cod
Siiue}) by taking the first three characters from that course. So a course with
code \texttt{INF7185\_1} will be from the department \texttt{INF}.

The sheet presents the first and last register of an activity in the Moodle,
and also the interval in days from the first activity to the last. The fields
are \texttt{min}, \texttt{max}, and \texttt{int}.

\TODO{Fields in between}

The sheet finally specifies the number of students who enrolled in a course and
the number of students who were approved on that course.

\begin{itemize}
    \item \texttt{course} - Course's id
    \item \texttt{Cod Moodle} - Course's Code in the Moodle
    \item \texttt{Cod Siiue} - Course's Code
    \item \texttt{min} - First activity of course
    \item \texttt{max} - Last activity of course
    \item \texttt{int} - Interval between first and last activity of course
    \item \texttt{actsub}
    \item \texttt{actcons}
    \item \texttt{nactsubd}
    \item \texttt{consal}
    \item \texttt{consdoc}
    \item \texttt{subal}
    \item \texttt{subdoc}
    \item \texttt{nalu}
    \item \texttt{ndoc}
    \item \texttt{InscSIIUE} - Number of students who enrolled
    \item \texttt{Aprov} - Number of students who got approved
    \item \texttt{Dep} - Department of the course
\end{itemize}

\subsection{Sheet 7 - \texttt{Correlação}}

This is a sheet that implements the calculation of correlation between the
variables of the previous sheet \texttt{ResultadosUCMoodle}. The data it self
is coppied from the previous sheet and a new table to the right of the first
table is added with the correlation itself.

The data in here doesn't need to be preprocessed to a new sheet, but the notion
of correlation, which can be calculated using alternative methods, may be used
later.

\subsection{Sheet 8 - \texttt{Alunos SIIUE E-L}}

There are a few issues with this sheet. First of all, the sheet appears to have
two tables which are not really related, maybe? Anyway. The first table has a
student number, his name, and what appears to be a count of it's enrollments.
It also has a calculated fields stating his/her's first and last name and
another column with the students number which is a calculated copy of the
first.

The second table has Moodle usernames. A first and last name, another count
(which I assume is a count of enrollments but I can't be sure because the field
simply states ``count'').

Oh! The second number in the first table is used for a calculated value in the
second table.

So the last three columns of the second table are all calculated values. The
first appears to fetch a number from the first. It goes like this. If the name
of the student in the second table is equal to the first and last name of the
first, then fetch that number. Else, it is an error.

Next, we get a value, is the username in the Moodle contains a number. If not,
we get an error.

Lastly. We try to get the first calculated number (the fourth column of the
second table) if it exists. If it doesn't, we try to get the second calculated
value (the fifth column of the second table). If that doesn't exist too, we
write 0.

Up to this point I'm not sure what the purpose of this sheet is. I believe it
gives me a relation between a student in SIIUE and in Moodle, but I have
serious doubts about the correctiveness of that relation.

After examining the sheet a second, I now believe that the first table has a
listing o students and their SIIUE internal id or student number. The second
table simply has usernames of Moodle. So the second table has students and
professors, while the first one has students only.

About the counts. The count of the first table is indeed a number of courses
where the students is enrolled in. The count of the second table is the total
number of log entries which is a copy (a copy-paste values only copy) of the
information in the \texttt{AlunosMoodle} sheet.

I don't think this sheet alone is much useful. But it did give insight on how
this data repository is organized.

\subsection{Sheet 9 - \texttt{AlunosMoodle}}

These are the actual logs of the accesses to Moodle. This is a big table that
contains the following fields:

\begin{itemize}
    \item \texttt{c\_id} - Course ID in the Moodle
    \item \texttt{c\_cod} - Course ID
    \item \texttt{c\_role} - Role of the user
    \item \texttt{c\_user} - User's code
    \item \texttt{c\_nome} - User's name
    \item \texttt{c\_uid} - User's ID in the Moodle
    \item \texttt{crud} - \gls{crud} activity
    \item \texttt{semana} - Number of the week in which the events happened
    \item \texttt{n} - Number of time the event happened on that week
    \item \texttt{numAluno} - Number of the student, which is calculated from
        elsewhere, even when the user is not a student
    \item \texttt{SNot} - If the student was approved, or failed, or something
        else
\end{itemize}

Basically, we get a list of entries specifying activities in Moodle. Each entry
contains information on the course, the user, and the activity itself.

For the course information we have course id and course code (Moodle code).

For user, we get the user's role, username, actual name of user, and some
number.

Finally, we get what kind of CRUD activity it was (if create activity, delete,
etc), the week in which said activity occurred, and number of times the
activity occurred in that week. The activities are aggregated by type and week.

This table then contains the fields \texttt{numAluno} and \texttt{SNot}. These
are calculate fields. The first attempts to take a student's number from the
username, if possible. The second calculates if that particular student was
approved in the course for which the entries refers to. Both of these fields
are useless.

There is a second table. This table gives a username, actual name, and total
number of activities of that user in the log. It is useful, but it should be in
it's own sheet. The fields of that table have no name, but in order to be
consistent with the table on the left, we get:

\begin{itemize}
    \item \texttt{c\_user}
    \item \texttt{c\_nome}
    \item \texttt{count}
\end{itemize}

\subsection{Sheet 10 - \texttt{CalculosAlunos}}

I'm not sure what his sheet is. It has a few values copied from other tables,
has ``\textit{Calculos}'', on the name, but it doesn't really calculate
anything. All the values are literal.

It may be discarded.

\subsection{Sheet 11 - \texttt{export\_weekly}}

This table seems to have a line for each pain of course and user ids. It then
has a few columns of log activities. For each intersection of pair of course
and user ids we get a 1 or 0 for one of the columns. Presumably, we get a true,
that is, 1, if a user A made that activity in course B in that week.

This isn't really useful because the same can be easily calculated using other
tools when needed.

\subsection{Sheet 12 - \texttt{export\_activity}}

This table seems to have a few more calculations, but I don't fully understand
them. It has at least a calculation of a standard deviation, and little else.

Again, this isn't really useful because the same can be easily calculated using
other tools when needed.

% =============================================================================
\section{Objects, Entities, and Features}

After a first analysis of the data, the objects in the data are identified.
These objects are the courses, students, results, etc. Data mining tasks are
applied over these objects which are described by their features.

This section specifies objects in this data repository and states their
features.

\subsection{Courses}

In the data, a course is identified by one of three unique ids. They are, the
Course ID, SIIUE's Course's Code, and Moodle Course's Code. The Course id is
simply a unique positive integer number which is probably assigned as an
incremental value when the courses where created in some database.

The two course's code come from their register in the SIIUE system and in the
Moodle system. For the most part they are the same, when they are not the same
the differences are minimal.

Table~\ref{tab:course_code_example} shows an example of the three ids. Notice how \texttt{CourseCodeMoodle} is either the same as \texttt{CourseCodeSiiue} or has just minimal differences (in this case there is an underscore follwed by a number).

\begin{table}[h!]
    \centering

    \begin{tabular}{l l l}
\texttt{CourseId} & \texttt{CourseCodeSiiue} & \texttt{CourseCodeMoodle} \\ \hline
        496       & \texttt{ARC10548}        & \texttt{ARC10548}         \\
        1392      & \texttt{GES7182}         & \texttt{GES7182\_2}       \\
        79        & \texttt{GES7182}         & \texttt{GES7182\_4}       \\
        1535      & \texttt{GES7182}         & \texttt{GES7182\_3}       \\
    \end{tabular}

    \caption
        [Example of course id and codes in the \texttt{CoursesGeneral} Dataset]
        {Example of course id and codes in the \texttt{CoursesGeneral} Dataset.}

    \label{tab:course_code_example}
\end{table}

Each course has the features listed in
table~\ref{tab:courses_general_features}. The table displayed the features and
their data type. Each course has a name, a department to which they belong, a
cycle (if Master's Degree, Phd, etc), and a degree. These features are all
strings.

A course also has a regime, a semester, a season, and a type. These features
are enumerated types because they have a limited domain. For example, the
semester may either be ``\textit{Par}'' or ``\textit{Ímpar}'' (Even or Odd),
the season may be ``\textit{Normal}'' or ``\textit{Special}''.

Finally, the credits are the only field which is a number.

\begin{table}[h!]
    \centering

    \begin{tabular}{l l}
        Feature             & Data Type  \\ \hline
        \texttt{Department} & String     \\
        \texttt{Cycle}      & String     \\
        \texttt{CourseName} & String     \\
        \texttt{Regime}     & Enumerated \\
        \texttt{Credits}    & Number     \\
        \texttt{Degree}     & String     \\
        \texttt{Semestre}   & Enumerated \\
        \texttt{Season}     & Enumerated \\
        \texttt{Type}       & Enumerated \\
    \end{tabular}

    \caption
        [Data types of features of Courses]
        {Data types of features of Courses as they appear in the
        \texttt{CoursesGeneral} dataset.}

    \label{tab:courses_general_features}
\end{table}

Some courses have no Moodle. They are therefore useless for our these
experiments.

Some statistics are taken from the courses data as shown in
table~\ref{tab:courses_stats}.

The domain of the enumerated fields are listed in table~\ref{tab:courses_enum}.
The field ``type'' only has the domain ``\textit{Normal}'', so it is a field
which is useless for further analysis. The field ``Semester'' has a missing
values. In the actual data there is only one course which doesn't have a value
for this field.

\begin{table}[h!]
    \centering

    \begin{tabular}{| l | r |}
        \hline
        Number of Courses     & 77 \\ \hline
        Number of Departments & 9  \\ \hline
        Number of Degrees     & 11 \\ \hline
    \end{tabular}

    \caption
        [Statistics on the Courses]
        {Statistics on the Courses.}

    \label{tab:courses_stats}
\end{table}

\begin{table}[h!]
    \centering

    \begin{tabular}{| l | l |}
        \hline
        Regime   & \{ A, O, S \}                                    \\ \hline
        Semester & \{ ``\textit{Par}'', ``\textit{Ímpar}'', n/a \}  \\ \hline
        Season   & \{ ``\textit{Normal}'', ``\textit{Especial}'' \} \\ \hline
        Type     & \{ ``\textit{Normal}'' \}                        \\ \hline
    \end{tabular}

    \caption
        [Values for enumerated types on courses]
        {Values for enumerated types on courses.}

    \label{tab:courses_enum}
\end{table}

\subsection{Moodle Users - Students, and Professors}

The logs contain the actions of the Moodle users which role can be of a
professors or a students. The data contains features for Students but no
features for professors. Table~\ref{tab:students_features}, contains all the
features of students.

\begin{table}[h!]
    \centering

    \begin{tabular}{l l}
        Feature                   & Data Type  \\ \hline
        \textit{StudentsId}       & Number     \\
        \textit{StudentsNumber}   & Number     \\
        \textit{MoodleUsername}   & String     \\
        \textit{Name}             & String     \\
        \textit{EnrollementCount} & Number     \\
        \textit{StudentType}      & Enumerated \\
    \end{tabular}

    \caption
        [Student features]
        {Student features.}

    \label{tab:students_features}
\end{table}

From table~\ref{tab:students_features} it is observable that a student may be
identified by it's internal Moodle Student ID, or by it's student number as
given by SIIUE. Some students in this data repository do not have a
\textit{StudentsNumber} value, but those students also don't have any activity
in the logs.

The Students Number has a leading character indicating the cycle of that
student. For example, a Master's Degree student has a number of the form
\textit{m12345}, starting with an \textit{m}, and a PhD student has a number
\textit{p12345} starting with a \textit{p}. In the data, the values for
\textit{StudentsNumber} are typed as number and only contain the number part.
But the field \textit{MoodleUsername} has the complete student's number.

The values for the \textit{StudentType} field are:

\begin{itemize}
    \item \textit{Normal}
    \item \textit{Interno}
\end{itemize}

Professors have only the features listed in table~\ref{tab:professor_features}.
The data contains a field called \textit{Role} which states if that Moodle user
is a professor or a student. Supposedly, if the user is a professor the role
has a value of 5. If it is a student, it has a value of 3. However, we see
entries for professors with a role value of 3 and vice versa. It is unknown if
this role actually state the if a user is a professor or student or something
else.

\begin{table}[h!]
    \centering

    \begin{tabular}{l l}
        Feature                 & Data Type \\ \hline
        \textit{MoodleUsername} & String    \\
        \textit{Name}           & String    \\
    \end{tabular}

    \caption
        [Professor features]
        {Professor features.}

    \label{tab:professor_features}
\end{table}

Table~\ref{tab:moodle_users_stats} shows some stats regarding Moodle users.

\begin{table}[h!]
    \centering

    \begin{tabular}{| l | r |}
        \hline
        Number of Students   & 145 \\ \hline
        Number of Professors & 87  \\ \hline
        Total                & 232 \\ \hline
    \end{tabular}

    \caption
        [Professor features]
        {Professor features.}

    \label{tab:moodle_users_stats}
\end{table}

\subsection{Results}

The data repository contains the results of students in the courses where they
participated. A listing of grades students had in particular courses is
available. Each object of that dataset, referred to as a Result, has the
features listing in table~\ref{tab:results_fetures}. The values for the
enumerated types are in table~\ref{tab:results_fetures_enum}.

\begin{table}[h!]
    \centering

    \begin{tabular}{l l}
        Feature                  & Data Type  \\ \hline
        \textit{CourseCodeSiiue} & String     \\
        \textit{StudentsNumber}  & Number     \\
        \textit{Grade}           & Number     \\
        \textit{Results}         & Enumerated \\
        \textit{FinalResult}     & Enumerated \\
    \end{tabular}

    \caption
        [Results features]
        {Results features.}

    \label{tab:results_fetures}
\end{table}

\begin{table}[h!]
    \centering

    \begin{tabular}{| l | l |}
        \hline
        Results & \{ \textit{Anulado},
                     \textit{Aprovado},
                     \textit{Desistiu},
                     \textit{Faltou},
                     \textit{Reprovado}
                  \} \\ \hline
        FinalResult & \{ N, S, n/a \}  \\ \hline
    \end{tabular}

    \caption
        [Values for enumerated types on results]
        {Values for enumerated types on results.}

    \label{tab:results_fetures_enum}
\end{table}

A student has a grade if he was approved (Results field has a value of
\textit{Aprovado}). The value for \textit{FinalResult} is \textit{S} if and
only if the value of \textit{Results} is \textit{Aprovado}. Else, it will be
\textit{N}.

The fields \textit{Results} and \textit{FinalResult} are always empty together
and in that cases there is never any grade.

Table~\ref{tab:results_stats} shows some statistics on the results data.

\begin{table}[h!]
    \centering

    \begin{tabular}{| l | r |}
        \hline
        Number of results           & 716 \\ \hline
        Number of non-empty results & 57  \\ \hline
        Number of approved students & 334 \\ \hline
    \end{tabular}

    \caption
        [Statistics on the Results]
        {Statistics on the Results.}

    \label{tab:results_stats}
\end{table}

\subsection{Moodle Logs}

Moodle logs contain the actual activities of Moodle users. Each log entry
states a user who performed a~\gls{crud} action in a course page. The actions
are aggregated per week. Table~\ref{tab:moodle_logs_features} shows the
features of the Moodle logs, and table~\ref{tab:moodle_logs_enum} shows the
values of the enumerated fields. There are a total of 25207 entries in the
Logs.

\begin{table}[h!]
    \centering

    \begin{tabular}{l l}
        Feature                            & Data Type  \\ \hline
        \textit{CourseCodeMoodle}          & String     \\
        \textit{Role}                      & Enumerated \\
        \textit{MoodleUsername}            & String     \\
        \textit{StudentsId}                & Number     \\
        \textit{CRUD}                      & Enumerated \\
        \textit{Week}                      & Number     \\
        \textit{NumberOfActivitiesPerWeek} & Number     \\
        \textit{StudentsNumber}            & Number     \\
    \end{tabular}

    \caption
        [Moode Logs features]
        {Moode Logs features.}

    \label{tab:moodle_logs_features}
\end{table}

\begin{table}[h!]
    \centering

    \begin{tabular}{| l | l |}
        \hline
        Role & \{ 3, 5 \} \\ \hline
        CRUD & \{ C, R, U, D \}  \\ \hline
    \end{tabular}

    \caption
        [Values for enumerated types on Moodle Logs]
        {Values for enumerated types on Moodle Logs.}

    \label{tab:moodle_logs_enum}
\end{table}

% =============================================================================
\section{Preprocessed Datasets}

From the original data we make an initial pre processing in order to have a
consistent repository of datasets ready for further analysis. The pre-processed
datasets have consistent names and fields.

% CoursesGeneral.csv
% CoursesEntriesOverview.csv
% MoodleUsers.csv
% Students.csv
% CoursesStudents.csv
% Results.csv
% MoodleLogs.csv
% CoursesNoMoodle.csv
