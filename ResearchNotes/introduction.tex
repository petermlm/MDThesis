\chapter{Introduction}

This document contains comments made to several papers which relate Data Mining
done to Moodle or other~\gls{lms} systems. The document is done in the scope of
a project where data from the Moodle of the University of Évora is mined. The
state of the art of that project is done from research described in this
document.

Each paper in this document has a general comment about it, a comment about
what are the more relevant points from it, and a comment regarding further work
from this paper. The general comments about the paper are structured as a five
paragraph essay.

The five paragraph essay follow a structure of the form:

\begin{itemize}
    \item Introduction
    \item Exposure
    \item Arguments for
    \item Arguments against
    \item Conclusions
\end{itemize}

Each paper is identified by the code \texttt{IND\_XXX}, where \texttt{XXX} is a
number with 3 leading zeros. Each paper is also identified by a reference in
the references section. As some papers reference each other, these references
are kept in the comments. The reason to identify a particular paper with a code
is because it makes it easy to store them in disk due to their long names.

Note that the number in \texttt{IND\_XXX} doesn't imply an order in the papers
listed in the document. Also, just because there is a paper in
\texttt{IND\_904} and a paper \texttt{IND\_906}, that doesn't imply that a
paper \texttt{IND\_905} needs to exists in the document.
