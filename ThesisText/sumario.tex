\begin{tueSUMARIO}

E-Learning tem vindo a ganhar popularidade como forma de transmitir educação
devido aos avanços nas tecnologias, como a Internet. Instituições como
universidades e empresas têm vindo a a usar E-Learning para a transmissão de
educação para locais remotos estendendo o seu alcance a estudantes e
colaboradores que estão fisicamente distantes.

Sistemas chamado ``Learning Management Systemas'', como o Moodle, existem para
organizar E-Learning. Eles oferecem plataformas online onde professores e
educadores podem publicar conteúdo, organizar actividades, fazer avaliações, e
afins, de modo a que estudantes possam aprender e serem avaliados.

Estes sistemas geram e guardam muitos dados relacionados não só com o seu uso,
mas também relacionados com notas de estudantes. Este tipo de dados são
frequentemente chamados de Dados Educacionais. Métodos de Data Mining são
aplicados a estes dados de modo a fazer suposições não triviais. As técnicas
aplicadas tomam inspiração de projectos semelhantes no campo da Data Mining
Educacional. Este campo consiste na aplicação de métodos de Data Mining a Dados
Educacionais.

Neste projecto, um repositório de dados do Moodle da Universidade de Évora é
explorado. Técnicas de aprendizagem supervisionada são aplicadas aos dados de
modo a mostrar como é possível prever o sucesso de estudantes a partir do seu
uso do Moodle. Métodos de aprendizagem não supervisionada são também aplicados
de modo a mostrar como há divisões nos dados.

\end{tueSUMARIO}
