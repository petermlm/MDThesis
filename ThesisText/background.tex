\chapter{Background}

% =============================================================================
\section{E-Learning and Learning Management Systems}
\label{sec:elearning_and_lms}

Recently, E-Learning has been rising in popularity as a solution for the
delivery of training. Institutions such as universities and companies have
successfully implemented E-Learning to bring course content and evaluation to
students, and to deliver training to employees remotely. This has been made
possible by the advances in technology and the increasing use of the
Internet.

E-Learning is usually provided over the Internet using platforms
called~\gls{lms}. Examples of~\gls{lms} include Moodle, Blackboard Learn, D2L,
among others. These platforms organize users by separating them into roles such
as Professors and Students or Authors and Learners. The Professors or Authors
are the users who organize courses in universities or training in companies,
create and publish content, perform evaluation of students, etc. Students or
Learners are the users which actually perform the training and follow courses
in the platforms.~\cite{ind_010, ind_011, ind_013}

In the specific case of Moodle, content is divided into courses. Each Moodle
course represents a course from a University or similar learning institution.

The contents of a course may be resources such as PDF files, links, media such
as images, videos, etc. These resources should be guides for students achieve
their objectives in learning, or should be resources needed for the student to
perform certain tasks. A course may also have activities. Activities may be,
for example, projects that a student needs to complete before a given deadline.
Activities may also be quizzes made to students over Moodle. These quizzes may
be used for evaluation.

Course also have forums, in which professors and students can exchange messages
which are relevant to the topics at hand.

A course is overseen by a group of Professors who will publish resources and
create activities. Students will use the resources and perform the
activities.~\cite{ind_014, ind_015}

% =============================================================================
\section{Data in Learning Management Systems}
\TODO{This section should just be part of the previous one.}

A typical~\gls{lms} contains various data entries which are stored in a
database. Information on users, such as their names, usage logs, roles, forum
messages, are generated by the systems and kept in their databases.

Resources and activities, as well as the logs detailing their usage, are also
kept in a database. The usage of these systems by professors and students
generates entries.

% =============================================================================
\section{Data Mining}

Current computational systems produce and store large amounts of data. This
data may represent anything, such as the sales records in a large commercial
chain, the usage telemetry on a popular web service, or even the registers of
some scientific survey. Data is usually rich in features. For example, the data
from the sales records would contain information on which items were sold and
in what quantity, to whom they were sold, on what date, and so on.

Data Mining aims at finding patterns in data by looking at the different
features of the data entries. These patterns allows then to make non trivial
assumptions in data.~\cite{book_dm_practical}

To find these patterns we build and train model such as Decision Trees,
Bayesian Networks, etc, from the data.

Data is usually stored in databases which may follow different paradigms and
have different architectures. But generally, we refer to data entries as
datasets. We also refer to a group of datasets as data repository.

\TODO{Talk about specific models}

% =============================================================================
\section{Educational Data Mining}

\TODO{Talk about clustering (See ind\_007 for things about this)}

Educational Data Mining, \gls{edm}, refers to the act of applying Data Mining
techniques to educational data. Educational data refers to data which origin is
in the act of teaching and training, in the activities of professors and
students, results of evaluation, and so on. In this project, the origin of data
is the University of Évora's Moodle.

The main objective of~\gls{edm} is to extract meaningful information from
educational data like web logs. It is not trivial to do so without advance Data
Mining techniques. \gls{edm}, therefore, aims at training models based on
education data for a variety of tasks.

One such example of~\gls{edm} Data Mining tasks is too predict student grades
based features from the students themselves, the courses they are undertaking,
and their activity in a~\gls{lms}. By training such models, professors can make
decisions about how courses are organized regarding students success. For
example, a certain number of activities in a given course may be beneficial to
it's success because it keeps students engaged, but any number of activities
more might not make a difference.~\cite{ind_001, ind_002, ind_005}

These models also detect students which may have greater difficulty completing
certain courses, or be used to predict how many students will complete a
certain class.~\cite{ind_007, ind_008}.

In~\cite{ind_001}, some models are trained in order to predict student grades.
The tasks described in the paper are classification tasks, in which a student
is assigned to a given class corresponding to their success in courses based on
their usage of a~\gls{lms} platform. The classes available are Excellent, Good,
Pass, and Fail.

The dataset used in~\cite{ind_001} contained the fields in
table~\ref{tab:ind_001_004_fields} and was made from registers of 438 students
in 7 different courses.

\begin{table}[h!]
    \centering

    \begin{tabular}{l l}
        Field name                       & Description                                    \\ \hline
        \texttt{course}                  & Identification number of the course.           \\
        \texttt{n\_assigment}            & Number of assignments done.                    \\
        \texttt{n\_quiz}                 & Number of quizzes taken.                       \\
        \texttt{n\_quiz\_a}              & Number of quizzes passed.                      \\
        \texttt{n\_quiz\_s}              & Number of quizzes failed.                      \\
        \texttt{n\_posts}                & Number of messages sent to the forum.          \\
        \texttt{n\_read}                 & Number or messages read on the forum.          \\
        \texttt{total\_time\_assignment} & Total time used on assignments.                \\
        \texttt{total\_time\_quiz}       & Total time used on quizzes.                    \\
        \texttt{total\_time\_forum}      & Total time used on forum.                      \\
        \texttt{mark}                    & Final mark the student obtained in the course. \\
    \end{tabular}

    \caption
        [Features in experiment~\cite{ind_001} and~\cite{ind_004}]
        {Features in experiment~\cite{ind_001} and~\cite{ind_004}.}

    \label{tab:ind_001_004_fields}
\end{table}

The used models are Statistical Classifiers, Decision Trees, Association Rules,
and Neural Networks. After training, performance measures are calculated using
test data. The best models were found to be decision trees according to the
calculated global percentage of correctly classified students.

The same experiment described in~\cite{ind_001} is also described
in~\cite{ind_004} with the same models, data, and results.

Similar models to before are used in~\cite{ind_005}. It is observed again that
Decision Trees have the best performance. Table~\ref{tab:ind_005_fields} shows
the data from~\cite{ind_005}. The experiments were made on data from 824
students in 11 courses.

\begin{table}[h!]
    \centering

    \begin{tabular}{l l}
        Field name                & Description                                              \\ \hline
        \texttt{UserName}         & Name of User                                             \\
        \texttt{CourseName}       & Name of the Course                                       \\
        \texttt{ResourceView}     & Number of Coursware and Other Supporting Materials Views \\
        \texttt{VirtualClassroom} & Number of Virtual Classroom Participations               \\
        \texttt{ArchiveView}      & Number of Archive Views                                  \\
        \texttt{ForumRead}        & Number of Forum Reads                                    \\
        \texttt{ForumPost}        & Number of Forum Posts                                    \\
        \texttt{DiscussionRead}   & Number of Discussion Reads                               \\
        \texttt{DiscussionPost}   & Number of Discussion Responses                           \\
        \texttt{AssignmentView}   & Number of Assignments Views                              \\
        \texttt{AssignmentUpload} & Number of Assignment Answer Uploads                      \\
        \texttt{FinalGrade}       & Final Grades                                             \\
    \end{tabular}

    \caption
        [Features in experiment~\cite{ind_005}]
        {Features in experiment~\cite{ind_005}.}

    \label{tab:ind_005_fields}
\end{table}

Neural networks are used in~\cite{ind_003} and~\cite{ind_006} for similar
purposes. In~\cite{ind_003} a neural network is used on a dataset with 116
entries. Each entry is made from 25 features and represents the results of
placements tests made to students. There are 5 outputs to the network, one for
each possible grade. The paper claims correctly classified rates of over 90\%
in some trained networks while most are just above 80\%. This shows how neural
networks can be used in this type of problems.

In~\cite{ind_006}, networks are trained from Moodle logs. The data from the
logs are structured in a slightly different way then the previous studies. The
features of this data are shown in table~\ref{tab:ind_006_fields}.

\begin{table}[h!]
    \centering

    \begin{tabular}{l}
        Description \\ \hline
        Full name of the student. \\
        Number of times that has been officially registered in the subject. \\
        Number of examination sessions. \\
        Mark in numerical format. \\
        Total of accesses (of any type) made to the Moodle system during course 2005/2006. \\
        Total of accesses to “resource view” \\
        Percentage of “resource view” accesses from the total accesses. \\
        Number of different “resource view”of each type (theoretical, examples, etc.) have been visited. \\
        Percentage of the resources of each type (theoretical, examples, etc.) sights. \\
        Segmentation of the number of accesses in every month of the year. \\
        Segmentation per month of the percentage of accesses. \\
    \end{tabular}

    \caption
        [Features in experiment~\cite{ind_006}]
        {Features in experiment~\cite{ind_006}.}

    \label{tab:ind_006_fields}
\end{table}

The data is extracted from 240 students. The number of courses is not
specified. Some trained networks are able to achieve 80\% of correctly
classified instances, while most achieve a value over 70\%.

% =============================================================================
\section{Data Mining Tools}

The practical component of this project was developed entirely using Python and
some Python libraries which are relevant in. The version of CPython used was
version 3.5.2. The python libraries Pandas, version 0.19.2, was used for
statistical analysis, exploration, and preprocessing. The library SciKit-Learn,
version 0.18.1, was used to perform Data Mining tasks. SciKit-Learn uses Numpy
and Scikit, in versions 1.11.3 and 0.18.1 respectively. To output some graphs,
the python library Mat Plot Lib, version 1.5.4, was used.

\TODO{State the actual methods and algorithms in SciKit-Learn that were used}
