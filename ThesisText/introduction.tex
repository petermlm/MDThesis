\chapter{Introduction}

In the last few years E-Learning has been rising in popularity as a way to
deliver education and training in a remote setting. Institutions such as
universities and companies have been taking advantage of E-Learning in
different ways. Universities have been developing remote learning courses in
order to reach more people, and companies have been developing remote training
solutions to reach people across different physical locations.

This rise in popularity of E-Learning is due to the some factors. The
advancements of technology, such as the Internet, have not only contributed to
the development of better and more complex systems, but have also made possible
for those systems to reach more people and obtain greater adoption from users
and institutions.

These E-Learning platforms are called~\gls{lms}. The platforms
allow institutions not only to deliver training but also keep track of users,
courses, content, etc. In their normal operations, theses platforms generate
and store a lot of data. This data concerns the courses, the content in those
courses, activities from users, and so on. In some specific cases, like Moodle
in universities, the system may even keep track of the student's grades.

The data becomes complex because these systems store a lot of information
regarding the users, courses, access logs, etc. Analysing this data, beyond
performing some statistical analysis, is not a trivial task. Some Data Mining
techniques may be applied in order to make less trivial assumptions.

The field of study which deals with applying~\gls{dm} techniques to educational
data is called~\gls{edm}. This projects applies~\gls{edm} to data from the
Moodle of the University of Évora.~\cite{ind_010, ind_011, ind_013, ind_014,
ind_015}

In this document we start by exploring the background of~\gls{dm} and,
specifically,~\gls{edm} in section~\ref{sec:back}. The background section also
talks about some~\gls{lms} and the tools used in this project.

Before applying those techniques, we explore the original datasets of the
Moodle provided for this specific project. Section~\ref{sec:data} talks about
the original datasets, how they were explored, and how they were preprocessed
into a data repository for use in this project.

Some statistical analysis are made to the preprocessed data repository before
any~\gls{dm} techniques are applied. This is done in order to understand the
contents of the data and know which~\gls{dm} techniques should be applied and
to which subset of the data repository those techniques should be applied to.
Section~\ref{sec:stats} talks about these statistics and draws some conclusions
from them.

Section~\ref{sec:exps} talks about the Data Mining techniques which are applied
to the data and also draws some conclusions from them. Those techniques
reference the statistics made in the previous section.

Finally, section~\ref{sec:final} draws some conclusions.
