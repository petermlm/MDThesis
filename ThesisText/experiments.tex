\chapter{Experiments}
\label{sec:exps}

The Data Mining experiments made in this project consist in organizing the data
from the processed datasets described in section~\ref{sec:preproc_dataset} into
a new dataset useful for a set of experiments. This chapter is organized by
sections on each dataset. Each dataset is used in several experiments with
different algorithms.

\section{Students, Courses, Logs, and Results}
\label{sec:exp_001_002}

In an initial analysis, we look into how Moodle usage relates with student's
success. A dataset is build with the features presented in
table~\ref{tab:dat_001}. The new dataset takes information from the following
datasets:

\begin{itemize}
    \item \texttt{Students}
    \item \texttt{CoursesGeneral}
    \item \texttt{MoodleLogs}
    \item \texttt{Results}
\end{itemize}

Each object of the dataset contains information referencing a single student in
a single course. The objects contain the overall enrollment count and type of
student. They also contain the identification of a course along with the
semester and season of that course. Finally, the object contains a count
of~\gls{crud} activities of that student in that course.

The dataset can be generated with different classes. There are three options.
Option one, the class is binary and has the value true if the student was
approved on that course, and false otherwise. Option two, the class has one of
the following values which are intervals of grades:

\[
    [0, 10[\;,\, [10, 12[\;,\, [12, 14[\;,\, [14, 16[\;,\, [16, 18[\;,\, [18, 20]
\]

Option three, the class has one of the following values:

\[
    [0, 10[\;,\, [10, 14[\;,\, [14, 18[\;,\, [18, 20]
\]

With option two, we get a dataset which groups every records with grades below
10, but separates the positive grades in groups of two. Option three does the
same but separates grades into groups of four, with the last group having only
three values instead of four.

\begin{table}[h!]
    \centering

    \begin{tabular}{| l | l | l |}
        \hline
        \textbf{Field}             & \textbf{Source Dataset}    & \textbf{Type} \\ \hline
        \texttt{EnrollementCount}  & Student                    & Number        \\ \hline
        \texttt{StudentType}       & Student                    & Enumerated    \\ \hline
        \texttt{CourseCodeSiiue}   & CoursesGeneral             & String        \\ \hline
        \texttt{Semester}          & CoursesGeneral             & Enumerated    \\ \hline
        \texttt{Season}            & CoursesGeneral             & Enumerated    \\ \hline
        \texttt{TotalActsOnCourse} & Calculated From MoodleLogs & Number        \\ \hline
        \texttt{CActsOnCourse}     & Calculated From MoodleLogs & Number        \\ \hline
        \texttt{RActsOnCourse}     & Calculated From MoodleLogs & Number        \\ \hline
        \texttt{UActsOnCourse}     & Calculated From MoodleLogs & Number        \\ \hline
        \texttt{DActsOnCourse}     & Calculated From MoodleLogs & Number        \\ \hline
        \texttt{Class}             & Calculated From Results    & Binary        \\ \hline
    \end{tabular}

    \caption
        [Students, Courses, Logs, and Results]
        {Students, Courses, Logs, and Results. This dataset joins information
        from four processed datasets in order to relate a student, a course,
        usage data from that student on that course, and the final grade of it}

    \label{tab:dat_001}
\end{table}

\subsection{Predicting Students Grades Based on Moodle Usage}
% EXP_001

A first experiment on this dataset aims at predicting the students grades based
on their Moodle usage data and in their other characteristics present on this
dataset. In this experiment six tasks are made. Both Decision Trees and Bayesian
Networks are trained using the dataset with binary classes, classes with
intervals of two, and classes with intervals of four.

Some fields on dataset~\ref{tab:dat_001} are strings and enumerated types,
therefore, these are nominal fields. Because of this characteristic, Decision
Trees and Bayesian Networks are used.

The models are trained using cross-validation with 10 splits. This means that
the data is divided into 10 parts. Nine parts are used to train to models and
the remaining part is used to test the models.

Training is executed 10 times, each time with different splits for training and
testing. The accuracy of each model is taken. Table~\ref{tab:exp_001_res} shows
the maximum accuracy of the trained models, the minimum, and the average.

\begin{table}[h!]
    \centering

    \begin{tabular}{| l | l | r | r | r |}
        \hline
        \textbf{Model} & \textbf{Classes} & \textbf{Maximum} & \textbf{Minimum} & \textbf{Average} \\ \hline
        Decision Tree    & Binary Class          & 0.92 & 0.34 & 0.70 \\
                         & Grades Classes 4 by 4 & 0.84 & 0.22 & 0.59 \\
                         & Grades Classes 2 by 2 & 0.74 & 0.34 & 0.55 \\ \hline
        Bayesian Network & Binary Class          & 0.95 & 0.53 & 0.74 \\
                         & Grades Classes 4 by 4 & 0.73 & 0.36 & 0.57 \\
                         & Grades Classes 2 by 2 & 0.67 & 0.28 & 0.52 \\ \hline
    \end{tabular}

    \caption
        [Predicting students grades based on Moodle usage results]
        {Predicting students grades based on Moodle usage results. Two models
        are trained 10 times on three different variations of the same dataset.
        The maximum, minimum, and average accuracy is displayed for those ten
        executions.}

    \label{tab:exp_001_res}
\end{table}

It is observable that for binary classes, Bayesian Networks performed slightly
better then decision trees. But for grade classes two by two and four by four,
decision trees were better. In particular, training models for binary classes,
understandably, yields better results on average.

These results show that such models can be trained to predict grades of
students with success.

\subsection{Clustering Students Grades Based on Moodle Usage}
% EXP_002

With the same dataset, clustering using the K-Means algorithm is performed,
however, the class feature is removed. The training and testing method is
similar to the previous experiment. Cross validation is used and a model is
trained 10 times, each time using nine different folds for training and one
fold for testing.

K-Means is trained and tested for a number of clusters between two and five.
Testing is done using silhouette score. Like before, the maximum, minimum, and
average score is taken. Table~\ref{tab:exp_002_res} shows the results.

\begin{table}[h!]
    \centering

    \begin{tabular}{| l | r | r | r |}
        \hline
        \textbf{Number of Clusters} & \textbf{Maximum} & \textbf{Minimum} & \textbf{Average} \\ \hline
        2 & 0.72 & 0.68 & 0.69 \\ \hline
        3 & 0.67 & 0.65 & 0.66 \\ \hline
        4 & 0.68 & 0.64 & 0.66 \\ \hline
        5 & 0.67 & 0.65 & 0.66 \\ \hline
    \end{tabular}

    \caption
        []
        {}

    \label{tab:exp_002_res}
\end{table}

In general, the model fails to get good results having a maximum score of 0.72
for two clusters. Numbers greater them five clusters were tried but they,
understandably, got lower scores.

To understand what the algorithm is separating, that is, which type of objects
are being placed in which cluster, we look in detail to an execution of K-Means
with two clusters.

Table~\ref{tab:exp_002a_res} has a listing of objects that are placed in each
of the two clusters. It is observable that the major difference is in the
number of~\gls{crud} activities. Different values for activities remain the
same for different courses and enrollment numbers, but greater values are
placed in one cluster while lower values are placed in another.

Besides K-Means, Affinity Propagation was executed with this dataset, but the
algorithm performed poorly have scores below 0.5.

\section{Number of Courses and Logs}

In similar experiments to~\ref{sec:exp_001_002}, we make a dataset in which we
relate a students total number of enrolled courses,  number of approved
courses, and Moodle usage data.

Each object of this dataset will have information on a single student. As
mentioned, an object has the number of enrolled and approved courses and the
sum of each~\gls{crud} activity for every course of that single student. Each
entry does not have the identification of the student to which it refers too
because that information is irrelevant for our learning objectives.

This dataset only contains features which have a number type. The records come
from the \texttt{Results} and \texttt{MoodleLogs} datasets.
Table~\ref{tab_dat_002} shows the features of this dataset.

\begin{table}[h!]
    \centering

    \begin{tabular}{| l | l | l |}
        \hline
        \textbf{Field}           & \textbf{Source Dataset}    & \textbf{Type} \\ \hline
        \texttt{EnrolledCourses} & Calculated From Results    & Number        \\ \hline
        \texttt{ApprovedCourses} & Calculated From Results    & Number        \\ \hline
        \texttt{ActSumAll}       & Calculated From MoodleLogs & Number        \\ \hline
        \texttt{ActSumC}         & Calculated From MoodleLogs & Number        \\ \hline
        \texttt{ActSumR}         & Calculated From MoodleLogs & Number        \\ \hline
        \texttt{ActSumU}         & Calculated From MoodleLogs & Number        \\ \hline
        \texttt{ActSumD}         & Calculated From MoodleLogs & Number        \\ \hline
    \end{tabular}

    \caption
        [Number of Courses and Logs]
        {Number of courses and logs. Each object in this dataset contains the
        number of enrolled courses, number of approved courses and Moodle usage
        data for a single student.}

    \label{tab:dat_002}
\end{table}

Some classification tasks use the \texttt{ApprovedCourses} feature as a class. In this case, feature may be left in it's original form or it may be replaced by classes. There are three ways to do this replacement.~\TODO{Explain this and mentioned statistics}

\subsection{Predicting Number of Approved Courses Based on Moodle Usage}
% EXP_003

Like before both Decision Trees and Bayesian Networks are used.
Table~\ref{tab:exp_003_res} shows the maximum, minimum, and average accuracy
for each pair of model and classes.

The models are trained using cross validation with 10 splits like before.

\begin{table}[h!]
    \centering

    \begin{tabular}{| l | l | r | r | r |}
        \hline
        \textbf{Model} & \textbf{Classes} & \textbf{Maximum} & \textbf{Minimum} & \textbf{Average} \\ \hline
        Decision Tree    & 2 Classes        & 1.00 & 0.64 & 0.83 \\
                         & 4 Classes        & 1.00 & 0.29 & 0.68 \\
                         & 6 Classes        & 1.00 & 0.21 & 0.68 \\
                         & Original Classes & 0.86 & 0.14 & 0.52 \\ \hline
        Bayesian Network & 2 Classes        & 1.00 & 0.47 & 0.70 \\
                         & 4 Classes        & 0.87 & 0.29 & 0.56 \\
                         & 6 Classes        & 0.86 & 0.07 & 0.43 \\
                         & Original Classes & 0.86 & 0.00 & 0.39 \\ \hline
    \end{tabular}

    \caption
        [Predicting number of approved courses based on Moodle usage results]
        {Predicting number of approved courses based on Moodle usage results.
        These are the results of two models being trained 10 times for 4
        different variants of the same dataset. The maximum, minimum, and
        average accuracy is shown.}

    \label{tab:exp_003_res}
\end{table}

It is observable that Decision Trees reached an accuracy of 1 producing, no
false negatives or false positives in testing, for all datasets except for the
one with the original classes. On average, Decision Trees, models achieve high
levels of accuracy for every dataset.

Bayesian Networks are achieve hight level of accuracy, but not as high as
Decision Trees. For the dataset with the original classes, Bayesian Networks
perform similarly in the best case scenario, given that the values for maximum
accuracy are similar. But the values for minimum and average are noticeably
lower.

Because the models trained with the original classes achieve high values for
accuracy we conclude that it is possible to predict the exact number of
approved courses with great accuracy given only the initial number of enrolled
courses and the Moodle usage data, especially if such tasks are executed with
Decision Trees.

\subsection{Clustering Number of Approved Courses Based on Moodle Usage}
% EXP_004

Clustering methods are applied to look for groups of similar objects within
this data. In this experiment, every field is a feature, so the
\texttt{ApprovedCourses} field is a feature and not a class like before.

\begin{table}[h!]
    \centering

    \begin{tabular}{| l | r | r | r |}
        \hline
        \textbf{Number of Clusters} & \textbf{Maximum} & \textbf{Minimum} & \textbf{Average} \\ \hline
        2 & 0.83 & 0.80 & 0.81 \\ \hline
        3 & 0.78 & 0.75 & 0.76 \\ \hline
        4 & 0.78 & 0.57 & 0.73 \\ \hline
        5 & 0.68 & 0.55 & 0.61 \\ \hline
        6 & 0.57 & 0.55 & 0.56 \\ \hline
    \end{tabular}

    \caption
        [Clustering number of approved courses based on Moodle usage results]
        {Clustering number of approved courses based on Moodle usage results.
        The table displays the results for the silhouette score for clustering.
        The values for every variant of the dataset are approximately the same,
        so only one table is shown.}

    \label{tab:exp_004_res}
\end{table}

Table~\ref{tab:exp_004_res} shows the results of this experiment. The
silhouette score shown is approximately the same for each variant of the
dataset, so only one table is shown for every variant.

It is observable that for two clusters, the score is around 0.8, and for three
clusters the score is around 0.75. This shows how these objects are able to be
separated in groups of two or three without having those groups overlapping.

For a number of clusters equal to or greater than four, we start getting scores
bellow 0.7.

The results shown were obtain with the K-Means algorithm. Affinity Propagation
was also executed, but the results obtained were not good, having scores within
0.0 and 0.5.

\section{Student/Course Profile and Results}

In this dataset we relate the profiling of a student in a courses and his
results on that course. To begin constructing this dataset we use the Profiling
dataset, as shown in section~\ref{sec:profile}.

The Profiling dataset contains a results field, however, the meaning of each
value in this field is unknown. So initially, this field is dropped. To get the
actual results of students we need the Results dataset.

Each object of Profile contains the features \texttt{CourseCodeMoodle},
\texttt{CourseId}, and \texttt{StudentsId}. But the Results dataset has
\texttt{CourseCodeSiiue} and \texttt{StudentsNumber}. In order to relate both,
the Students dataset and the CoursesGeneral dataset are needed.

These datasets are related and some fields are dropped in order to produce a
dataset with the features listed in table~\ref{tab:dat_003}.

Like in dataset~\ref{sec:exp_001_002}, the class of this dataset may be binary, in
which the student is approved or not, or divided in groups of 2 or 4.

\begin{table}[h!]
    \centering

    \begin{tabular}{| l | l | l |}
        \hline
        \textbf{Field}               & \textbf{Source Dataset} & \textbf{Type} \\ \hline
        \texttt{CourseCodeSiiue}     & CoursesGeneral          & String        \\ \hline
        \texttt{NumberOfResources}   & Profile                 & Number        \\ \hline
        \texttt{NumberOfActivities}  & Profile                 & Number        \\ \hline
        \texttt{NumberOfViews}       & Profile                 & Number        \\ \hline
        \texttt{NumberOfSubmissions} & Profile                 & Number        \\ \hline
        \texttt{Class}               & Results                 & Binary        \\ \hline
    \end{tabular}

    \caption
        [Student/Course Profile and Results]
        {Student/Course Profile and Results.}

    \label{tab:dat_003}
\end{table}

\subsection{Predicting Students Grades From Student/Course Profiling}
% EXP_005

Using the described dataset, we train models to predict student grades based on
profiling data. Table~\ref{tab:exp_005_res} shows the results of six tasks.
Three for each Model. Each tasks is done in a variant of the dataset.

\begin{table}[h!]
    \centering

    \begin{tabular}{| l | l | r | r | r |}
        \hline
        \textbf{Model} & \textbf{Classes} & \textbf{Maximum} & \textbf{Minimum} & \textbf{Average} \\ \hline
        Decision Tree    & Binary Class          & 0.91 & 0.43 & 0.73 \\
                         & Grades Classes 4 by 4 & 0.65 & 0.31 & 0.44 \\
                         & Grades Classes 2 by 2 & 0.56 & 0.18 & 0.34 \\ \hline
        Bayesian Network & Binary Class          & 0.97 & 0.49 & 0.77 \\
                         & Grades Classes 4 by 4 & 0.91 & 0.31 & 0.55 \\
                         & Grades Classes 2 by 2 & 0.71 & 0.26 & 0.46 \\ \hline
    \end{tabular}

    \caption
        [Predicting students grades from student/course profiling results]
        {Predicting students grades from student/course profiling results. The
        results show the maximum, minimum, and average for each trained model
        and each dataset variant.}

    \label{tab:exp_005_res}
\end{table}

We observe from the results that Bayesian Networks perform better then Decision
Trees achieving greater levels of accuracy for each dataset variant.

\subsection{Clustering Students Grades From Student/Course Profiling}
% EXP_006

Taking the same dataset, we remove the field \texttt{CourseCodeSiiue} and apply
clustering like in previous experiments. Table~\ref{tab:exp_006_res} shows the
results of this experiment.

\begin{table}[h!]
    \centering

    \begin{tabular}{| l | r | r | r |}
        \hline
        \textbf{Number of Clusters} & \textbf{Maximum} & \textbf{Minimum} & \textbf{Average} \\ \hline
        2 & 0.64 & 0.60 & 0.63 \\ \hline
        3 & 0.58 & 0.56 & 0.57 \\ \hline
        4 & 0.57 & 0.53 & 0.54 \\ \hline
        5 & 0.51 & 0.50 & 0.50 \\ \hline
    \end{tabular}

    \caption
        []
        {}

    \label{tab:exp_006_res}
\end{table}

Having two clusters we get the higher values for silhouette score, but those
values are still far from 1, meaning that the executed algorithms do not find
any meaningful separation of objects into groups.
