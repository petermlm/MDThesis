\begin{tueABSTRACT}

E-Learning as been rising in popularity as a way to deliver training due to the
advancements of technologies, like the Internet. Institutions such as
universities and companies have been making use of E-Learning to deliver
training to remote locations extending their reach to students and employees
who are physically distant.

Systems called Learning Management Systems, like Moodle, exist to organize
E-Learning. They provide online platforms where professors and educators can
publish content, organize activities, perform evaluations, and so on, in order
for students to learn and get evaluated.

These systems generate and store lots of data regarding not only their usage,
but also regarding the grades of students. This kind of data is often referred
too as Educational Data. Data Mining techniques are applied to this data in
order to make non trivial assumptions. The techniques applied take inspiration
from similar projects within the field of Educational Data Mining. This field
consists in applying Data Mining Techniques to Education Data.

In this project, a data repository from the Moodle of the University of Évora
is explored. Supervised learning techniques are applied to this data in order
to show how it is possible to make predictions about the success of students
based on their usage of Moodle. Unsupervised learning techniques are also
applied in order to show how data is divisible.

\end{tueABSTRACT}
