\chapter{Statistics on the Data}
\label{sec:stats}

Before applying any Data Mining methods, it is important to understand how the
data is constituted. Knowing things like, how many courses are there, how many
students, how many students per course, etc, is important in order to train
models and interpret the results.

\section{Courses}

To begin exploring the courses some counts are made. We count the number of
overall courses, the number of departments, and number of degrees.
Table~\ref{tab:courses_stats} shows these counts. We see 77 courses which are
part of 9 different departments and make up 11 degrees. While the number of
courses is much greater then some experiments mentioned in the background, it
is important to understand how many students does each of these courses have.

\begin{table}[h!]
    \centering

    \begin{tabular}{| l | r |}
        \hline
        Number of Courses     & 77 \\ \hline
        Number of Departments & 9  \\ \hline
        Number of Degrees     & 11 \\ \hline
    \end{tabular}

    \caption
        [Courses in numbers]
        {Courses in numbers.}

    \label{tab:courses_stats}
\end{table}

If some courses only have one or two students who fail, they might not be
useful for training of the model because they won't add any useful information
to them. To understand the distribution of students in the courses, we plot the
number of students for each course, and the number of approved students.
Figure~\ref{fig:stat_007} shows that plot. Each bar is a course, and courses
are sorted by total number of students.

\begin{figure}[h!]
    \centering

    \includegraphics[width=\linewidth]{../src/stat_007_results/plot}

    \caption
        [Number of Students per Course and Number of Approved Students per
        Course]
        {Number of Students per Course and Number of Approved Students per
        Course. The grey bars show the number of students, while the blue bars
        show the number of approved students. The graph}

    \label{fig:stat_007}
\end{figure}

Courses have different values for credits. Table~\ref{tab:courses_credits}
shows how many courses are there for each number of credits. We see that the
majority of courses has 6 credits.

\begin{table}[h!]
    \centering

    \begin{tabular}{| l | r |}
        \hline
        Credits & Number of Courses \\ \hline
        1       & 3                 \\ \hline
        2       & 3                 \\ \hline
        3       & 3                 \\ \hline
        5       & 6                 \\ \hline
        6       & 46                \\ \hline
        7.5     & 12                \\ \hline
        12      & 2                 \\ \hline
        15      & 2                 \\ \hline
    \end{tabular}

    \caption
        [Number of courses group by credits]
        {Number of courses group by credits.}

    \label{tab:courses_credits}
\end{table}

We take the number of courses for each degree. Table~\ref{tab:stat_003_res}.
The actual name of the degrees is not available, so we can only identify them
by their codes. Looking into the counts we see that the degree with code
\texttt{B\_M\_REST\_EINF\_E (578)} has the majority of courses by far. From
this table we also see that some degrees have a low number of degrees and some
even have have courses only on one semester.

It was to be expected that every degree would have a similar number of courses,
or at least that every course have curses for both semesters. It is unknown why
there are degrees present that have such characteristics, whether it is just
missing data or if there really are such degrees.

\begin{table}[h!]
    \centering

    \begin{tabular}{| l | l | r | r |}
        \hline
        Degree                             & Semester       & Number of Courses & Total \\ \hline
        \texttt{B\_M\_EHEA (478)}          & \textit{Par}   & 3                 & 7     \\
                                           & \textit{Ímpar} & 4                 &       \\ \hline
        \texttt{B\_M\_REST\_EINF\_E (578)} & \textit{Par}   & 14                & 28    \\
                                           & \textit{Ímpar} & 14                &       \\ \hline
        \texttt{B\_M\_REST\_TET (457)}     & \textit{Par}   & 2                 & 7     \\
                                           & \textit{Ímpar} & 5                 &       \\ \hline
        \texttt{B\_PD\_M\_E (580)}         & \textit{Par}   & 3                 & 8     \\
                                           & \textit{Ímpar} & 5                 &       \\ \hline
        \texttt{PG\_B\_EGNEG (554)}        & \textit{Par}   & 4                 & 9     \\
                                           & \textit{Ímpar} & 5                 &       \\ \hline
        \texttt{PG\_B\_AE (438)}           & \textit{Par}   & 5                 & 10    \\
                                           & \textit{Ímpar} & 5                 &       \\ \hline
        \texttt{FC\_EDM (564)}             & \textit{Par}   & 1                 & 1     \\ \hline
        \texttt{CCDND\_CPM (344)}          & \textit{Ímpar} & 1                 & 1     \\ \hline
        \texttt{FC\_CVP\_FDOC (513)}       & \textit{Ímpar} & 2                 & 2     \\ \hline
        \texttt{FC\_CVP\_UAM (514)}        & \textit{Ímpar} & 3                 & 3     \\ \hline
    \end{tabular}

    \caption
        [Number of Courses per Degree and Semester]
        {Number of Courses per Degree and Semester.}

    \label{tab:stat_003_res}
\end{table}

\section{Students}

Table~\ref{tab:moodle_users_stats} shows some stats regarding Moodle users.
Some of the explored projects listed in the background, like~\TODO{cite}, have
a number of students around 800 to 900 which is greater then the number of
students in this dataset, but they also have a lower number of courses, for
example~\TODO{cite} has 7, where this dataset has 77.

\begin{table}[h!]
    \centering

    \begin{tabular}{| l | r |}
        \hline
        Number of Students   & 145 \\ \hline
        Number of Professors & 87  \\ \hline
        Total                & 232 \\ \hline
    \end{tabular}

    \caption
        [Professor features]
        {Professor features.}

    \label{tab:moodle_users_stats}
\end{table}

Observing the behaviour of students in most courses in any University, it is
common to find that the number of students decreases over time has the course
progresses. This is due to students giving up on that course for that
particular semester for various reasons. Although the data doesn't have any
explicit data point stating that a student gave up on a course, we can still
observe student activity to verify if it decreases or not as the semester
progresses over the weeks.

The activity on a course is measured using the Moodle logs which contain the
CRUD activities. CRUD activities, as stated in~\ref{sec:logs}, are aggregated
by courses, students, and weeks. To have a sense of how activity changes over a
course we take the logs dataset and aggregate every object only by course. So,
each entry will have the sum of the activities aggregated by course and week.

To get a sense of activities over the semester, we take six courses. Two of the courses are the ones with the greatest number of activities, other two are the ones with a median number of activities, and the last two are courses with the least number of activities.

We take the two courses with the most activity overall and plot a graph which x
axis is the weeks and y axis is the number of activities.
Figure~\ref{fig:stat_004_all} shows the graph for these courses. The number of
read activities and number of total activities is much greater then the number
of create, update, and delete activities. Because of this, we plot a graph
which only contains create, update, and delete activities for the same courses
as shown in figure~\ref{fig:stat_004_cud}.

\begin{figure}[h!]
    \centering

    \begin{subfigure}{.5\textwidth}
        \centering
        \includegraphics[width=\linewidth]{../src/stat_004_results/fig_most_0_CTE-UMCED-2_all}
        \caption[]{}
        \label{subfig:stat_most_004_0_all}
    \end{subfigure}%
    \begin{subfigure}{.5\textwidth}
        \centering
        \includegraphics[width=\linewidth]{../src/stat_004_results/fig_most_1_eDocOn-T1_all}
        \caption[]{}
        \label{subfig:stat_most_004_1_all}
    \end{subfigure}

    \begin{subfigure}{.5\textwidth}
        \centering
        \includegraphics[width=\linewidth]{../src/stat_004_results/fig_middle_0_MAT7514_1_all}
        \caption[]{}
        \label{subfig:stat_middle_004_0_all}
    \end{subfigure}%
    \begin{subfigure}{.5\textwidth}
        \centering
        \includegraphics[width=\linewidth]{../src/stat_004_results/fig_middle_1_GES11656_all}
        \caption[]{}
        \label{subfig:stat_middle_004_1_all}
    \end{subfigure}

    \begin{subfigure}{.5\textwidth}
        \centering
        \includegraphics[width=\linewidth]{../src/stat_004_results/fig_least_0_HIS10755_all}
        \caption[]{}
        \label{subfig:stat_least_004_0_all}
    \end{subfigure}%
    \begin{subfigure}{.5\textwidth}
        \centering
        \includegraphics[width=\linewidth]{../src/stat_004_results/fig_least_1_MAT7515_all}
        \caption[]{}
        \label{subfig:stat_least_004_1_all}
    \end{subfigure}

    \caption
        [Number of CRUD activities for courses]
        {Number of CRUD activities for two courses with greater activity
        (\ref{subfig:stat_most_004_0_all} and
        \ref{subfig:stat_most_004_1_all}), median activity
        (\ref{subfig:stat_middle_004_0_all} and
        \ref{subfig:stat_middle_004_1_all}), and least activity
        (\ref{subfig:stat_least_004_0_all} and
        \ref{subfig:stat_least_004_1_all}).}

    \label{fig:stat_004_all}
\end{figure}

\begin{figure}[h!]
    \centering

    \begin{subfigure}{.5\textwidth}
        \centering
        \includegraphics[width=\linewidth]{../src/stat_004_results/fig_most_0_CTE-UMCED-2_cud}
        \caption[]{}
        \label{subfig:stat_most_004_0_cud}
    \end{subfigure}%
    \begin{subfigure}{.5\textwidth}
        \centering
        \includegraphics[width=\linewidth]{../src/stat_004_results/fig_most_1_eDocOn-T1_cud}
        \caption[]{}
        \label{subfig:stat_most_004_1_cud}
    \end{subfigure}

    \begin{subfigure}{.5\textwidth}
        \centering
        \includegraphics[width=\linewidth]{../src/stat_004_results/fig_middle_0_MAT7514_1_cud}
        \caption[]{}
        \label{subfig:stat_middle_004_0_cud}
    \end{subfigure}%
    \begin{subfigure}{.5\textwidth}
        \centering
        \includegraphics[width=\linewidth]{../src/stat_004_results/fig_middle_1_GES11656_cud}
        \caption[]{}
        \label{subfig:stat_middle_004_1_cud}
    \end{subfigure}

    \begin{subfigure}{.5\textwidth}
        \centering
        \includegraphics[width=\linewidth]{../src/stat_004_results/fig_least_0_HIS10755_cud}
        \caption[]{}
        \label{subfig:stat_least_004_0_cud}
    \end{subfigure}%
    \begin{subfigure}{.5\textwidth}
        \centering
        \includegraphics[width=\linewidth]{../src/stat_004_results/fig_least_1_MAT7515_cud}
        \caption[]{}
        \label{subfig:stat_least_004_1_cud}
    \end{subfigure}

    \caption
        [Number of Create, Update, and Delete activities for courses]
        {Number of Create, Update, and Delete activities for two courses with
        greater activity (\ref{subfig:stat_most_004_0_cud} and
        \ref{subfig:stat_most_004_1_cud}), median activity
        (\ref{subfig:stat_middle_004_0_cud} and
        \ref{subfig:stat_middle_004_1_cud}), and least activity
        (\ref{subfig:stat_least_004_0_cud} and
        \ref{subfig:stat_least_004_1_cud}).}

    \label{fig:stat_004_cud}
\end{figure}

From figures~\ref{fig:stat_004_all} and~\ref{fig:stat_004_cud} we see that the
activities of courses remain generally the same during the semester. With the
exception of courses with the least activity, it is observable that while there
are a few peaks during the semester, there is no significant drop in activities
towards the end of them. The examples shown have peaks in various places during
the semester, however, they don't tend to be at the start or at the end of the
semester.

This leads to the conclusion not only that the levels of activity remain
relatively constant during the semester, but also leads to the conclusion that
many students don't quit the courses even i they fail.

To get an overall view of how activities change during the whole semester we
make another plot. This time, we see the total number of activities for every
course during the semester and the mean of activities.
Figure~\ref{fig:stat_005} shows this plot.

The number of total activities per week drops significantly. This is because
some courses are longer then others. For example, some courses may have their
last evaluation in the first weeks of the exam season, while others may still
have ongoing activity after exam season.

\TODO{Explain what are active courses and show the plot}

The mean of activities remains the same. The mean is calculated for each week
only on courses that have any activity on that week. So looking at the last few
weeks we get a mean of activities only for courses which are still active. With
this in mind, we again see that courses remain relatively the same in terms of
activities while the semester is going.

\begin{figure}[h!]
    \centering

    \includegraphics[width=\linewidth]{../src/stat_005_results/plot}

    \caption
        [Total Number of Activities vs Mean of Activities per Week]
        {Total Number of Activities vs Mean of Activities per Week.}

    \label{fig:stat_005}
\end{figure}

\section{Results}

Table~\ref{tab:results_stats} shows some statistics on the results data.

\begin{table}[h!]
    \centering

    \begin{tabular}{| l | r |}
        \hline
        Number of results           & 716 \\ \hline
        Number of non-empty results & 57  \\ \hline
        Number of approved students & 334 \\ \hline
    \end{tabular}

    \caption
        [Statistics on the Results]
        {Statistics on the Results.}

    \label{tab:results_stats}
\end{table}

\section{Profile}

\section{Moodle Logs}
