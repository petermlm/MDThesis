\chapter{University of Évora's Moodle Data}
\label{sec:data}

% =============================================================================
\section{Data Exploration}

To begin with, we explore the data which was initially provided. The repository
is a spreadsheet book with 12 sheets. The sheets contain raw data and some
calculated fields.

In order to explore the data using the bash tools (cut, grep, etc) and tools
like Python with Pandas, they were exported to a CSV format. Some
transformations were applied to these files in order to clean useless data
points or make the data easier to process.

The provided sheets contain information of courses, students, their activities
in Moodle, and their results. The data comes from Moodle databases and from the
SIIUE system. Both systems are integrated, making Moodle plus SIIUE the
University's own~\gls{lms}.

The sheets are the following:

\begin{itemize}
    \item \texttt{UCs\_SIIUE\_Moodle.csv}
    \item \texttt{complete\_rate.csv}
    \item \texttt{completed.csv}
    \item \texttt{Notas UCS SIIUE.csv}
    \item \texttt{Profile}
    \item \texttt{ResultadosUCMoodle}
    \item \texttt{Correlação}
    \item \texttt{Alunos SIIUE E-L}
    \item \texttt{AlunosMoodle}
    \item \texttt{export\_weekly}
    \item \texttt{export\_activity}
\end{itemize}

Some sheets contained information that was the result of calculations from the
established data. Analysis like these were removed from the data for two
reasons. First, some analysis are not useful so them being in the data only
make it more complex. By removing such analysis we get a simpler data
model. Second, even if some analysis are in fact useful, they are still easily
reproduced by the Python tools, with the difference that they are only
reproduced when needed for specific experiments.

The sheets that remained after removing analysis data are:

\begin{itemize}
    \item \texttt{UCs\_SIIUE\_Moodle.csv}
    \item \texttt{Notas UCS SIIUE.csv}
    \item \texttt{Profile}
    \item \texttt{ResultadosUCMoodle}
    \item \texttt{Alunos SIIUE E-L}
    \item \texttt{AlunosMoodle}
\end{itemize}

The rest of the sheets contains the objects which are going to be used in the
Data Mining tasks. Those objects and their features are fairly unorganized in
the original data, so an analysis on of the available features had to be done.
Section~\ref{sec:objs_feat} states the objects found in the data and details
their features. Section~\ref{sec:preproc_dataset} shows how the final dataset
is organized.

% =============================================================================
\section{Objects and Features}
\label{sec:objs_feat}

After a first analysis of the data, the objects are identified. These objects
are courses, students, results, etc. Data mining tasks are applied over these
objects which are described by their features.

This section specifies objects in this data repository and states their
features.

\subsection{Courses}
\label{sec:data_courses}

In the data, a course is identified by one of three unique ids. They are, the
Course Id, SIIUE's Course's Code, and Moodle Course's Code. The Course Id is
simply a unique positive integer number which is probably assigned as an
incremental value when the courses where created in some database.

The two course's code come from their register in the SIIUE system and in the
Moodle system. For the most part they are the same, when they are not the same
the differences are minimal.

Table~\ref{tab:course_code_example} shows an example of the three ids. Notice
how \texttt{CourseCodeMoodle} is either the same as \texttt{CourseCodeSiiue} or
has just minimal differences (in this case there is an underscore followed by a
number).

\begin{table}[h!]
    \centering

    \begin{tabular}{l l l}
\texttt{CourseId} & \texttt{CourseCodeSiiue} & \texttt{CourseCodeMoodle} \\ \hline
        496       & \texttt{ARC10548}        & \texttt{ARC10548}         \\
        1392      & \texttt{GES7182}         & \texttt{GES7182\_2}       \\
        79        & \texttt{GES7182}         & \texttt{GES7182\_4}       \\
        1535      & \texttt{GES7182}         & \texttt{GES7182\_3}       \\
    \end{tabular}

    \caption
        [Example of course id and codes in the \texttt{CoursesGeneral} Dataset]
        {Example of course id and codes in the \texttt{CoursesGeneral} Dataset.}

    \label{tab:course_code_example}
\end{table}

Each course has the features listed in
table~\ref{tab:courses_general_features}. The table displayed the features and
their data type. Each course has a name, a department to which they belong, a
cycle (if Master's Degree, Phd, etc), and a degree. These features are all
strings.

A course also has a regime, a semester, a season, and a type. These features
are enumerated types because they have a limited domain. For example, the
semester may either be ``\textit{Par}'' or ``\textit{Ímpar}'' (Even or Odd).
The type only has one possible value which is \texttt{Normal}. Despite this,
the field was kept in the data for completeness. Table~\ref{tab:courses_enum}
shows the domain of each enumerated type.

Finally, the credits are the only field which is a number.

\begin{table}[h!]
    \centering

    \begin{tabular}{l l}
        Feature             & Data Type  \\ \hline
        \texttt{Department} & String     \\
        \texttt{Cycle}      & String     \\
        \texttt{CourseName} & String     \\
        \texttt{Regime}     & Enumerated \\
        \texttt{Credits}    & Number     \\
        \texttt{Degree}     & String     \\
        \texttt{Semestre}   & Enumerated \\
        \texttt{Season}     & Enumerated \\
        \texttt{Type}       & Enumerated \\
    \end{tabular}

    \caption
        [Data types of features of Courses]
        {Data types of features of Courses as they appear in the
        \texttt{CoursesGeneral} dataset.}

    \label{tab:courses_general_features}
\end{table}

Some courses have no Moodle and, as such, have no other associated information.
They are therefore useless for these experiments and were simply removed.

\begin{table}[h!]
    \centering

    \begin{tabular}{| l | l |}
        \hline
        Regime   & \{ A, O, S \}                            \\ \hline
        Semester & \{ \textit{Par}, \textit{Ímpar}, n/a \}  \\ \hline
        Season   & \{ \textit{Normal}, \textit{Especial} \} \\ \hline
        Type     & \{ \textit{Normal} \}                    \\ \hline
    \end{tabular}

    \caption
        [Values for enumerated types on courses]
        {Values for enumerated types on courses.}

    \label{tab:courses_enum}
\end{table}

\subsection{Moodle Users - Students, and Professors}
\label{ref:moodle_users}

The logs contain the actions of the Moodle users which role can be of a
professor or a student. The data contains features for students but no
features for professors. Table~\ref{tab:students_features}, contains all the
features of students.

\begin{table}[h!]
    \centering

    \begin{tabular}{l l}
        Feature                   & Data Type  \\ \hline
        \textit{StudentsId}       & Number     \\
        \textit{StudentsNumber}   & Number     \\
        \textit{MoodleUsername}   & String     \\
        \textit{Name}             & String     \\
        \textit{EnrollementCount} & Number     \\
        \textit{StudentType}      & Enumerated \\
    \end{tabular}

    \caption
        [Student features]
        {Student features.}

    \label{tab:students_features}
\end{table}

From table~\ref{tab:students_features} it is observable that a student may be
identified by its internal Moodle Student ID, or by its student number as
given by SIIUE. Some students in this data repository do not have a
\textit{StudentsNumber} value, but those students also don't have any activity
in the logs.

The Students Number has a leading character indicating the cycle of that
student. For example, a Master's Degree student has a number of the form
\textit{m12345}, starting with an \textit{m}, and a PhD student has a number
\textit{p12345} starting with a \textit{p}. In the data, the values for
\textit{StudentsNumber} are typed as number and only contain the number part.
But the field \textit{MoodleUsername} has the complete student's number.

The Name field contains the actual name of the student. This information is
useless for any analysis, but the name was kept for completeness.

The values for the \textit{StudentType} field are \textit{Normal} or
\textit{Interno} (Normal or Internal).

Professors have only the features listed in table~\ref{tab:professor_features}.
The data contains a field called \textit{Role} which states if that Moodle user
is a professor or a student. If the user is a professor the role has a value of
5. If it is a student, it has a value of 3. A few users with a value of 4 are
present. These users have the role of editor, but the Moodle logs only have 11
entries for a user of this role, so they are simply discarded.

\begin{table}[h!]
    \centering

    \begin{tabular}{l l}
        Feature                 & Data Type \\ \hline
        \textit{MoodleUsername} & String    \\
        \textit{Name}           & String    \\
    \end{tabular}

    \caption
        [Professor features]
        {Professor features.}

    \label{tab:professor_features}
\end{table}

\subsection{Results}
\label{sec:results}

The data repository contains the results of students in the courses where they
participated. A listing of grades students had in particular courses is
available. Each object of that dataset, referred to as a Result, has the
features listing in table~\ref{tab:results_fetures}. The values for the
enumerated types are in table~\ref{tab:results_fetures_enum}.

\begin{table}[h!]
    \centering

    \begin{tabular}{l l}
        Feature                  & Data Type  \\ \hline
        \textit{CourseCodeSiiue} & String     \\
        \textit{StudentsNumber}  & Number     \\
        \textit{Grade}           & Number     \\
        \textit{Results}         & Enumerated \\
        \textit{FinalResult}     & Enumerated \\
    \end{tabular}

    \caption
        [Results features]
        {Results features.}

    \label{tab:results_fetures}
\end{table}

\begin{table}[h!]
    \centering

    \begin{tabular}{| l | l |}
        \hline
        Results & \{ \textit{Anulado},
                     \textit{Aprovado},
                     \textit{Desistiu},
                     \textit{Faltou},
                     \textit{Reprovado}
                  \} \\ \hline
        FinalResult & \{ N, S, n/a \}  \\ \hline
    \end{tabular}

    \caption
        [Values for enumerated types on results]
        {Values for enumerated types on results.}

    \label{tab:results_fetures_enum}
\end{table}

A student has a grade if he was approved (Results field has a value of
\textit{Aprovado}). The value for \textit{FinalResult} is \textit{S} if and
only if the value of \textit{Results} is \textit{Aprovado}. Else, it will be
\textit{N}.

The fields \textit{Results} and \textit{FinalResult} are always empty together
and in those cases there is never any grade.

\subsection{Profile}
\label{sec:profile}

Profile objects are objects that, for each pair of course and student, state
the following indicators:

\begin{itemize}
    \item \texttt{NumberOfResources}
    \item \texttt{NumberOfActivities}
    \item \texttt{NumberOfViews}
    \item \texttt{NumberOfSubmissions}
\end{itemize}

A resource in this context is any file or link that a professor may have placed
in a Moodle page. Profile objects state the number of resources which were seen
by a particular student in a particular course.

An activity is any interaction in a Moodle forum or quizzed. A view is a view
of any page, resource, etc. A submission is when a students submits a project.

There are a total of 331 entries in the Profile dataset.

\subsection{Moodle Logs}
\label{sec:logs}

Moodle logs contain the actual activities of Moodle users. Each log entry
states a user who performed a~\gls{crud} action in a course page. The actions
are aggregated per week. Table~\ref{tab:moodle_logs_features} shows the
features of the Moodle logs, and table~\ref{tab:moodle_logs_enum} shows the
values of the enumerated fields. There are a total of 25206 entries in the
Logs.

\begin{table}[h!]
    \centering

    \begin{tabular}{l l}
        Feature                            & Data Type  \\ \hline
        \textit{CourseCodeMoodle}          & String     \\
        \textit{Role}                      & Enumerated \\
        \textit{MoodleUsername}            & String     \\
        \textit{StudentsId}                & Number     \\
        \textit{CRUD}                      & Enumerated \\
        \textit{Week}                      & Number     \\
        \textit{NumberOfActivitiesPerWeek} & Number     \\
        \textit{StudentsNumber}            & Number     \\
    \end{tabular}

    \caption
        [Moodle Logs features]
        {Moodle Logs features.}

    \label{tab:moodle_logs_features}
\end{table}

\begin{table}[h!]
    \centering

    \begin{tabular}{| l | l |}
        \hline
        Role & \{ 3, 5 \} \\ \hline
        CRUD & \{ C, R, U, D \}  \\ \hline
    \end{tabular}

    \caption
        [Values for enumerated types on Moodle Logs]
        {Values for enumerated types on Moodle Logs.}

    \label{tab:moodle_logs_enum}
\end{table}

% =============================================================================
\section{Preprocessed Datasets}
\label{sec:preproc_dataset}

From the original data we make an initial preprocessing in order to have a
consistent repository of datasets ready for further analysis. The preprocessed
datasets have consistent names and fields.

The data stored in CSV files from which any statistics are calculated and
experiments are executed. The data is also organized following a relational
paradigm.~\footnote{Despite the fact that the data is not kept in a Relational
Database, having it organized in a relational paradigm simplifies the process
of working with it using the Python tools}

The following sections describe which datasets store the data of the various
objects as described in section~\ref{sec:objs_feat}. In the end of this
section, figure~\ref{fig:er} shows the relational model of the various datasets.

\begin{figure}[h!]
    \centering

    \includegraphics[width=\linewidth]{../ER/er.pdf}

    \caption
        [Relational model for processed data repository]
        {Relational model for processed data repository.}

    \label{fig:er}
\end{figure}

\subsection{Courses Datasets}

Data on the courses is store in the following dataset files:

\begin{itemize}
    \item \texttt{CoursesEntriesOverview}
    \item \texttt{CoursesGeneral}
    \item \texttt{CoursesNoMoodle}
    \item \texttt{CoursesStudents}
\end{itemize}

The first dataset contains mostly what is found in the original
\texttt{ResultadosUCMoodle} dataset. The second dataset contains general
information on each course, namely the three codes that identify each course,
the course name, regime, credits, etc.

The \texttt{CoursesNoMoodle} dataset simply contains a listing of courses that
have no moodle. These courses aren't very important since no other information
on them is available. But they are still kept in the processed datasets.

The \texttt{CoursesStudents} dataset contains one entry for each course. Each
entry states the total number of students of that course, the total number of
approved students, and the approval ration for that course. This information
would easily be calculated from other datasets in this repository, but it is
kept here since it was present in the original datasets as a non calculated
field.

\subsection{Moodle Users Datasets}

Each Moodle user has a username, an actual name, and a role. That information
is stored in \texttt{MoodleUsers}. This dataset contains users who are both
professors and students. They are identified by their role.

Students have the fields stated in section~\ref{ref:moodle_users}, and those
are stored in the \texttt{Students} dataset.

\subsection{Results}

Results are stored in the \texttt{Results} dataset. The results consist of the
fields stated in table~\ref{tab:results_fetures} in section~\ref{sec:results}.

\subsection{Profiling Datasets}

The \texttt{Profile} dataset contains the information stated in
section~\ref{sec:profile}. The fields are shown in
table~\ref{tab:profile_features}.

\begin{table}[h!]
    \centering

    \begin{tabular}{l l}
        Feature                      & Data Type \\ \hline
        \texttt{CourseCodeMoodle}    & String    \\
        \texttt{CourseId}            & String    \\
        \texttt{StudentsId}          & Number    \\
        \texttt{NumberOfResources}   & Number    \\
        \texttt{NumberOfActivities}  & Number    \\
        \texttt{NumberOfViews}       & Number    \\
        \texttt{NumberOfSubmissions} & Number    \\
    \end{tabular}

    \caption
        [Profile features]
        {Profile features.}

    \label{tab:profile_features}
\end{table}

\subsection{Moodle Logs Dataset}

The logs dataset in \texttt{MoodleLogs} contain the entries as described in
section~\ref{sec:logs}.
